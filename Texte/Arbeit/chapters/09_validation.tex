% ----------------------------------------------------------------------------
% Chapter 9: Validation  (Hinweis: ggf. in Kap. 8 integrieren)
% ----------------------------------------------------------------------------
%
% ============================================================
% QUELLENVORSCHLÄGE (falls Kapitel erhalten bleibt)
% ============================================================
% Physikalische Konsistenz / Inkompressibilitätsprüfung:
%   \parencite{richter-powell_neural_nodate}  -- Divergenzfreiheit durch psi-Repräsentation
%   \parencite{ribeiro_deepcfd_nodate}        -- Validierung gegen CFD-Referenz
%
% Robustheit / Unsicherheit:
%   \parencite{zhu_bayesian_2018}            -- Bayesianische Unsicherheitsschätzung für Surrogate
%   \parencite{morimoto_generalization_2022} -- Robustheitsbewertung über Geometrietypen
%
% Vergleich mit klassischen Löser-Ergebnissen:
%   \parencite{verhoeven_numerical_2009}     -- Numerische Referenz Kristallsedimentation
%   \parencite{patocka_settling_2020}        -- Weitere Referenz Sedimentation
%   \parencite{ladd_numerical_nodate}        -- LBM-Referenzlösungen
% ============================================================
%
\chapter{Validation}
\label{ch:validation}

Kann glaub ich gestrichen werden, da ich die Validierung ja schon in Kapitel 8 bespreche. In diesem Kapitel könnte ich vielleicht noch einmal kurz zusammenfassen, wie ich die Validierung durchgeführt habe und welche Ergebnisse ich dabei erzielt habe. Es könnte auch eine Diskussion darüber geben, wie robust meine Ergebnisse sind und ob es mögliche Einschränkungen oder Unsicherheiten gibt, die berücksichtigt werden sollten. Aber insgesamt denke ich, dass die meisten Informationen zur Validierung bereits in Kapitel 8 enthalten sind und dass dieses Kapitel vielleicht nicht unbedingt notwendig ist. --- IGNORE ---