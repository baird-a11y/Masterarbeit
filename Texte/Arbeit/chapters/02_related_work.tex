\chapter{State of the Art}
\label{ch:related}

This chapter reviews recent developments in machine-learning-based surrogate 
models for fluid dynamics, with a particular emphasis on incompressible Stokes 
flow, flow–structure interaction, and neural architectures operating on 
structured grids. While classical numerical methods remain the reference 
standard in terms of physical fidelity, their computational cost often renders 
large parameter studies infeasible. This limitation has motivated the growing 
interest in data-driven surrogate models that aim to approximate solution 
operators at a fraction of the computational expense.

The literature surveyed in this chapter provides the conceptual and 
methodological foundation for the modeling choices made in this thesis. In 
particular, it highlights both the opportunities and the limitations of 
existing surrogate approaches when applied to problems involving complex, 
geometry-dependent flow fields.

% ----------------------------------------------------------------------
\section{Deep Learning for Fluid Dynamics}

Deep neural networks have emerged as powerful tools for approximating solutions 
of partial differential equations, especially in fluid dynamics. Convolutional 
neural networks (CNNs) and encoder–decoder architectures such as the U-Net have 
proven particularly effective for mapping geometric input fields to structured 
solution fields \parencite{navab_u-net_2015,thuerey_deep_2020}. Their success can be 
attributed to a combination of local feature extraction, multiscale 
representation, and translation-invariant convolutional operators.

Early studies on CNN-based surrogates for steady laminar flows demonstrated 
that these models can reproduce high-fidelity Navier--Stokes or Stokes 
solutions with speed-ups of several orders of magnitude compared to 
traditional solvers, while maintaining acceptable accuracy 
\parencite{ribeiro_deepcfd_nodate,chen_u-net_nodate}. These results established deep 
learning as a viable surrogate modeling strategy for computational fluid 
dynamics.

Subsequent work extended these approaches to more complex geometries and 
boundary conditions. In particular, scalable U-Net variants for flow 
prediction in multi-obstacle environments showed that multiscale 
encoder–decoder architectures with skip connections are well suited for 
problems that require simultaneous resolution of sharp boundary layers and 
global flow structures \parencite{thuerey_deep_2020}. This combination of local and 
global context is directly relevant for multi-crystal sedimentation, where 
localized wake structures around individual crystals interact with 
domain-scale circulation patterns.

% ----------------------------------------------------------------------
\section{Physics-Informed and Hybrid Approaches}

Physics-Informed Neural Networks (PINNs) aim to incorporate physical knowledge 
directly into the learning process by augmenting the loss function with 
residuals of the governing partial differential equations 
\parencite{raissi_physics-informed_2019}. Rather than relying exclusively on paired 
input–output data, PINNs enforce approximate satisfaction of the underlying 
equations at collocation points within the computational domain. For 
incompressible flow, this typically involves penalizing the momentum equations 
together with the divergence-free constraint \parencite{jin_nsfnets_2021}.

Hybrid extensions of this idea include geometry-aware PINNs, which embed 
geometric information through distance functions or latent representations to 
improve learning on irregular domains \parencite{oldenburg_geometry_2022}. While 
these approaches can enhance physical consistency, they often introduce 
additional challenges. In practice, PINNs and hybrid formulations tend to be 
sensitive to the relative weighting of loss terms, exhibit slow convergence, 
and incur significant computational overhead due to repeated evaluation of 
spatial derivatives.

These difficulties become particularly pronounced for creeping Stokes flow 
with multiple rigid inclusions, where sharp gradients near particle boundaries 
coexist with long-range hydrodynamic interactions. For this reason, the 
present thesis does not adopt a full physics-informed formulation. Instead, 
physical structure is incorporated implicitly through the choice of target 
representation, most notably by predicting the stream function $\psi$, which 
enforces incompressibility by construction while retaining the efficiency of a 
purely data-driven training procedure.

% ----------------------------------------------------------------------
\section{Neural Operators and Generalization Across Geometries}

Neural operators seek to learn mappings between function spaces rather than
between individual solution instances. In contrast to classical surrogate
models, which approximate a solution for a fixed discretization or geometry,
neural operators aim to approximate the underlying solution operator itself.

A particularly prominent representative is the Fourier Neural Operator (FNO)
proposed by \textcite{li_fourier_2021}. The FNO parameterizes integral kernel
operators in the spectral domain by applying the Fast Fourier Transform,
learning a truncated set of Fourier coefficients at each layer. This
architecture offers two key advantages: first, it captures global, long-range
dependencies in a single layer—a property that is naturally suited for
problems governed by elliptic operators such as the Stokes equations; second,
it achieves discretization invariance, meaning that a model trained on one
grid resolution can, in principle, be evaluated on a different resolution
without retraining. FNOs have demonstrated strong performance on benchmark
PDE problems including Darcy flow, Burgers' equation, and two-dimensional
Navier--Stokes turbulence \parencite{li_fourier_2021}.

More broadly, convolution-based neural operators on structured grids have
demonstrated the ability to generalize across variations in material
parameters, forcing terms, and boundary conditions
\parencite{thuerey_deep_2020}. These properties make neural operators appealing
candidates for surrogate modeling in multiphase flow problems.

However, most existing studies focus on flows in relatively regular domains
or on smoothly varying geometries. Applications to systems with many
interacting rigid inclusions—such as sedimenting crystals with strong
hydrodynamic coupling—remain scarce. In particular, the question of whether
learned operators can generalize across changes in topological complexity,
such as varying particle numbers ($1 \ldots N$), has not been systematically
addressed. This gap is especially relevant for geophysical sedimentation
problems, where the number and spatial arrangement of crystals are inherently
variable.

% ----------------------------------------------------------------------
\section{Machine Learning for Particle- and Crystal-Laden Flows}

Fluid–particle interaction has long been an active area of research in 
classical computational fluid dynamics. Fully resolved numerical approaches, 
including lattice Boltzmann, immersed boundary, and fictitious-domain methods, 
have been used to study drafting–kissing–tumbling dynamics, clustering, and 
collective settling behavior in particulate flows 
\parencite{fortes_nonlinear_1987,ladd_numerical_1994,leonardi_coupled_2014}. These 
methods provide detailed physical insight but are computationally expensive, 
particularly as the number of particles increases.

Machine-learning-based surrogates for particle- or crystal-laden flows are 
only beginning to emerge. Existing studies primarily focus on learning reduced 
quantities, such as drag corrections, closure terms, or accelerations within 
coupled CFD–DEM frameworks. In contrast, surrogate models that directly map 
multi-particle geometric configurations to full flow fields remain largely 
unexplored, especially in the low-Reynolds-number regime characteristic of 
magmatic systems.

This lack of full-field surrogates represents a significant limitation for 
applications where detailed spatial information is required, for example to 
analyze wake interactions, boundary-layer structures, or the emergence of 
collective sedimentation patterns.

% ----------------------------------------------------------------------
\section{Comparison of Network Paradigms for Flow-Field Surrogates}

A wide range of machine-learning architectures has been proposed for 
approximating solutions of partial differential equations. For flow-field 
prediction problems, three paradigms are particularly relevant: 
coordinate-based implicit models, linear reduced-order models, and grid-based 
convolutional architectures.

\subsection{Coordinate-Based Implicit Models}

Coordinate-based neural networks represent physical fields as continuous 
functions of spatial coordinates. These models can achieve high accuracy and 
produce smooth reconstructions on fixed geometries. However, explicitly 
encoding complex and variable geometries within coordinate-based 
representations remains challenging. Extending such approaches to systems 
with multiple interacting inclusions typically requires substantial redesign 
of the input representation, limiting their scalability across varying 
particle numbers.

\subsection{Linear Reduced-Order Models}

Linear reduced-order models, such as Proper Orthogonal Decomposition or 
Principal Component Analysis, project high-dimensional flow fields onto a 
small number of dominant modes. While computationally efficient and 
interpretable, their linear nature fundamentally restricts their ability to 
capture nonlinear flow–geometry interactions. This limitation is particularly 
severe for multi-crystal Stokes flow, where wake interactions and boundary-layer 
effects play a central role.

\subsection{Grid-Based Convolutional Surrogates}

Grid-based convolutional neural networks operate directly on structured
meshes and are therefore well suited for problems in which both inputs and
outputs are defined on regular grids. U-Net architectures combine local feature
extraction with global context through their encoder–decoder structure and
skip connections \parencite{navab_u-net_2015}. A key advantage of this paradigm is
its geometry-agnostic input representation: masks and distance fields can
encode an arbitrary number of inclusions without requiring architectural
modifications.

Previous studies demonstrate that CNN-based surrogates can generalize
effectively across variable geometries and boundary conditions
\parencite{thuerey_deep_2020}. These properties make grid-based convolutional models
particularly attractive for multi-crystal sedimentation problems.

\subsection{Spectral Neural Operators}

Fourier Neural Operators \parencite{li_fourier_2021} constitute an alternative
grid-based paradigm that operates in the spectral rather than the spatial
domain. Instead of learning localized convolutional kernels, FNOs learn
global kernel functions parameterized by their Fourier coefficients. This
spectral perspective provides inherent access to long-range correlations
within a single network layer, which is advantageous for elliptic problems
where local perturbations propagate throughout the entire domain.

A potential limitation of the standard FNO architecture is its reliance on
low-frequency Fourier modes: the truncation of high-frequency coefficients
may reduce accuracy near sharp interfaces such as crystal boundaries, where
steep velocity gradients require high-frequency content. How this trade-off
affects predictive performance in multi-crystal Stokes flow is one of the
central questions investigated in this thesis.

\subsection{Implications for This Work}

The comparison of these paradigms reveals a trade-off between accuracy,
geometric flexibility, and scalability. Coordinate-based implicit models
excel on fixed geometries, and linear reduced-order models offer
interpretability, but neither scales naturally to variable multi-inclusion
configurations. Among grid-based approaches, U-Net and FNO architectures
represent two complementary strategies: U-Nets capture multiscale spatial
features through hierarchical encoder–decoder processing, while FNOs
capture global structure through spectral convolutions. Comparing these two
paradigms on the same multi-crystal Stokes flow problem constitutes a core
contribution of this thesis.

% ----------------------------------------------------------------------
\section{Position of This Thesis in the Research Landscape}

Based on the reviewed literature, four key gaps can be identified:

\begin{enumerate}
    \item The absence of surrogate models targeting multi-inclusion Stokes flow at the level of full flow fields.
    \item A lack of systematic evaluation of surrogate-model generalization with respect to particle number.
    \item Limited investigation of physically structured output representations—such as the stream function—as inductive biases in data-driven flow prediction.
    \item No direct comparison of spatial (U-Net) and spectral (FNO) architectures for surrogate modeling of particle-laden Stokes flow with variable geometry.
\end{enumerate}

This thesis addresses these gaps by constructing a dataset of multi-crystal
Stokes simulations, predicting the stream function $\psi$ as a physically
structured learning target, and systematically comparing U-Net and FNO
architectures with respect to generalization performance across varying
crystal numbers and geometric configurations.
