% -------------------------------------------------------
% Abstract (English, unnumbered)
% -------------------------------------------------------

\chapter*{Abstract}
\addcontentsline{toc}{chapter}{Abstract}

\begin{otherlanguage}{english}
Crystal sedimentation in viscous magmatic fluids is a fundamental geodynamic
process that drives chemical differentiation and controls the internal structure
of magma reservoirs. Accurate simulation of multi-crystal settling requires the
simultaneous resolution of microscale boundary layers around individual crystals
and long-range hydrodynamic interactions spanning the full computational domain.
This multi-scale character, combined with the near-quadratic scaling of
interaction costs with particle number, renders systematic parameter studies
with classical solvers computationally prohibitive.

This thesis investigates whether data-driven surrogate models can effectively
replace high-fidelity Stokes solvers for predicting flow fields in multi-crystal
sedimentation scenarios. Two neural network architectures are systematically
compared on identical data and evaluation metrics: the U-Net, an encoder--decoder
model with skip connections that extracts spatially hierarchical features, and
the Fourier Neural Operator (FNO), which operates in the spectral domain and
provides a global receptive field per layer. Both models are trained to predict
the stream function $\psi$ from a five-channel geometric input encoding consisting
of a crystal mask, a signed distance proxy field, the distance to the nearest
crystal boundary, and normalized spatial coordinates. Predicting $\psi$ rather
than velocity components directly enforces incompressibility by construction,
eliminating a common source of systematic constraint violations in purely
data-driven velocity surrogates.

Training data are generated using the \textsc{LaMEM} geodynamic finite-element
code, providing steady-state incompressible Stokes solutions for configurations
of $1$ to $N_{\max}$ randomly placed rigid circular crystals on a fixed
$256 \times 256$ grid. 

Siehe Anmerkung in der deutschen Zusammenfassung bezüglich genauer Zahlen, die ich noch einpflegen muss sobald ich die Versuche gemacht habe.


Gradient-derived velocity quantities prove consistently harder to
reproduce than $\psi$ itself, with relative errors roughly an order of magnitude
larger, reflecting the amplification of residuals under numerical differentiation.
Error diagnostics further show that prediction difficulty increases near crystal
boundaries and domain edges.

The controlled comparison of U-Net and FNO architectures on this benchmark
quantifies the trade-off between spatially local multiscale processing and global
spectral representations for surrogate modeling of elliptic, multi-inclusion
Stokes flow. The results provide practical guidelines for input encoding, output
representation, and architecture selection in geophysical sedimentation
applications.
\end{otherlanguage}
