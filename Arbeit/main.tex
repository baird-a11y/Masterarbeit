\documentclass[12pt,twoside,openright]{scrreprt}

% Grundlegende Pakete
\usepackage[utf8]{inputenc}
\usepackage[T1]{fontenc}
\usepackage[ngerman,english]{babel}
\usepackage{lmodern}
\usepackage{csquotes}

% Mathematik und Wissenschaft
\usepackage{amsmath,amssymb,amsthm,mathtools}
\usepackage{siunitx}
\usepackage{booktabs}

% Manuelle Definition häufig verwendeter Befehle
\newcommand{\grad}{\nabla}
\newcommand{\divg}{\nabla \cdot}
\newcommand{\curl}{\nabla \times}
\newcommand{\laplacian}{\nabla^2}

% Grafiken und Visualisierung
\usepackage{graphicx}
\usepackage{subcaption}
\usepackage{tikz}
\usepackage{pgfplots}
\pgfplotsset{compat=1.18}

% Code und Algorithmen
\usepackage{listings}
\usepackage{xcolor}
\usepackage{algorithm2e}

% Literatur und Verweise
\usepackage[style=authoryear,backend=biber,maxbibnames=3]{biblatex}
\addbibresource{references.bib}

% Layout-Anpassungen
\usepackage{geometry}
\geometry{a4paper,left=3cm,right=2.5cm,top=2.5cm,bottom=2.5cm}
\usepackage{setspace}
\onehalfspacing

% Hyperref sollte fast zuletzt geladen werden
\usepackage{hyperref}
\hypersetup{
	colorlinks=true,
	linkcolor=black,
	citecolor=blue,
	urlcolor=blue,
	pdftitle={Machine Learning für Multi-Kristall Strömungsfelder},
	pdfauthor={Paul Baselt}
}

% cleveref NACH hyperref laden
\usepackage{cleveref}

% Wissenschaftliche Formatierung
\usepackage{abstract}
\renewcommand{\abstractname}{Abstract}

% Theorem-Umgebungen
\theoremstyle{definition}
\newtheorem{definition}{Definition}[chapter]
\newtheorem{hypothesis}{Hypothesis}[chapter]

\theoremstyle{plain}
\newtheorem{theorem}{Theorem}[chapter]
\newtheorem{lemma}[theorem]{Lemma}

% Julia-Code Highlighting - VEREINFACHT
\lstdefinelanguage{Julia}{
	keywords={function,end,if,else,elseif,while,for,begin,let,try,catch,finally,return,break,continue,global,local,const,struct,mutable,abstract,primitive,type,where,using,import,export,module,baremodule,macro,quote,do},
	keywordstyle=\color{blue}\bfseries,
	sensitive=true,
	comment=[l]{\#},
	commentstyle=\color{gray},
	string=[b]",
	stringstyle=\color{red},
	morestring=[b]',
	basicstyle=\ttfamily\small,
	frame=single,
	numbers=left,
	numberstyle=\tiny\color{gray},
	breaklines=true,
	breakatwhitespace=true,
	tabsize=4,
	inputencoding=utf8,
	extendedchars=true,
	literate={η}{{$\eta$}}1 {Δ}{{$\Delta$}}1 {ρ}{{$\rho$}}1
}

% Spezielle Definitionen für Geowissenschaften
\newcommand{\reynolds}{\text{Re}}
\newcommand{\stokes}{\text{St}}
\newcommand{\peclet}{\text{Pe}}

% Titel und Metadaten (für PDF-Eigenschaften)
\title{Machine Learning zur Vorhersage von Strömungsfeldern um sinkende Kristalle: Generalisierungsfähigkeit von UNet-Architekturen für Multi-Kristall Sedimentationssysteme}
\author{Paul Baselt}
\date{\today}

\begin{document}

% PROFESSIONELLES TITELBLATT
\begin{titlepage}
    \centering
    
    % Logo der Universität (falls vorhanden)
    % \includegraphics[width=0.3\textwidth]{logo_uni_mainz.png}\\[1cm]
    
    % Universitätsinformationen
    {\large \textbf{Johannes Gutenberg-Universität Mainz}}\\[0.5cm]
    {\large Fachbereich 08 -- Physik, Mathematik und Informatik}\\[0.5cm]
    {\large Institut für Geowissenschaften}\\[2cm]
    
    % Titel der Arbeit
    {\huge \textbf{Machine Learning zur Vorhersage von Strömungsfeldern um sinkende Kristalle}}\\[0.5cm]
    {\LARGE \textbf{Generalisierungsfähigkeit von UNet-Architekturen für Multi-Kristall Sedimentationssysteme}}\\[1.5cm]
    
    % Art der Arbeit
    {\Large \textbf{Masterarbeit}}\\[0.5cm]
    zur Erlangung des akademischen Grades\\[0.3cm]
    {\large \textbf{Master of Science (M.Sc.)}}\\[0.3cm]
    in\\[0.3cm]
    {\large \textbf{Computational Sciences / Rechnergestützte Naturwissenschaften}}\\[2cm]
    
    % Verfasser
    vorgelegt von\\[0.5cm]
    {\large \textbf{Paul Baselt}}\\[0.3cm]
    Matrikelnummer: [Ihre Matrikelnummer]\\[0.3cm]
    aus [Ihr Geburtsort]\\[1.5cm]
    
    % Betreuung
    \begin{minipage}[t]{0.45\textwidth}
        \centering
        \textbf{Erstgutachter:}\\
        Univ.-Prof. Dr. Boris Kaus\\
        Institut für Geowissenschaften\\
        Johannes Gutenberg-Universität Mainz
    \end{minipage}
    \hfill
    \begin{minipage}[t]{0.45\textwidth}
        \centering
        \textbf{Zweitgutachter:}\\
        Prof. Dr. Michael Wand\\
        Institut für Informatik\\
        Johannes Gutenberg-Universität Mainz
    \end{minipage}\\[2cm]
    
    % Datum
    \vfill
    {\large Mainz, \today}
    
\end{titlepage}

% Leere Seite nach Titelblatt (für doppelseitigen Druck)
\newpage
\thispagestyle{empty}
\mbox{}

% Eidesstattliche Erklärung
\newpage
\thispagestyle{empty}

\vspace*{3cm}
\begin{center}
    {\Large \textbf{Eidesstattliche Erklärung}}
\end{center}

\vspace{2cm}

Hiermit erkläre ich, dass ich die vorliegende Masterarbeit selbständig verfasst und keine anderen als die angegebenen Quellen und Hilfsmittel verwendet habe. Alle Stellen, die wörtlich oder sinngemäß aus veröffentlichten oder unveröffentlichten Arbeiten entnommen sind, habe ich als solche kenntlich gemacht.

\vspace{1cm}

Die Arbeit ist in gleicher oder ähnlicher Form noch keiner anderen Prüfungsbehörde vorgelegt worden.

\vspace{3cm}

\noindent
\begin{tabular}{@{}p{7cm}p{7cm}@{}}
    \rule{6cm}{0.5pt} & \rule{6cm}{0.5pt} \\
    Ort, Datum & Unterschrift \\
\end{tabular}

% Abstrakt (Deutsch und Englisch)
\newpage
\begin{abstract}
Diese Masterarbeit untersucht die Generalisierungsfähigkeit von UNet-Architekturen zur Vorhersage von Strömungsfeldern in Multi-Kristall Sedimentationssystemen. Während traditionelle computational fluid dynamics (CFD) Methoden für komplexe Multi-Partikel Systeme rechenzeitintensiv sind, bieten machine learning Ansätze das Potenzial für erhebliche Beschleunigungen. Die vorliegende explorative Studie evaluiert systematisch, ob UNet-Netzwerke, die auf Ein-Kristall Konfigurationen trainiert wurden, erfolgreich auf Systeme mit 2-15 Kristallen generalisieren können. 

Mithilfe von LaMEM-Simulationen als Ground Truth werden kontrollierte Experimente zur Komplexitätssteigerung durchgeführt. Die Ergebnisse zeigen [Platzhalter für tatsächliche Resultate], wodurch sowohl die Möglichkeiten als auch die fundamentalen Limitationen von UNet-Generalisierung in Multi-Partikel Strömungssystemen charakterisiert werden. Diese Arbeit leistet einen methodischen Beitrag zur physics-informed machine learning Forschung und bietet praktische Erkenntnisse für geowissenschaftliche Modellierungsanwendungen.
\end{abstract}

\begin{otherlanguage}{english}
\begin{abstract}
This master's thesis investigates the generalization capability of UNet architectures for predicting flow fields in multi-crystal sedimentation systems. While traditional computational fluid dynamics (CFD) methods are computationally intensive for complex multi-particle systems, machine learning approaches offer the potential for significant acceleration.

This exploratory study systematically evaluates whether UNet networks trained on single-crystal configurations can successfully generalize to systems with 2-15 crystals. Using LaMEM simulations as ground truth, controlled experiments with increasing complexity are conducted. The results demonstrate [placeholder for actual findings], thereby characterizing both the possibilities and fundamental limitations of UNet generalization in multi-particle flow systems.

This work contributes methodologically to physics-informed machine learning research and provides practical insights for geoscientific modeling applications.
\end{abstract}
\end{otherlanguage}

% Verzeichnisse
\tableofcontents
\listoffigures
\listoftables

% Hauptteil
\chapter{Einleitung und Forschungsmotivation}
\label{ch:introduction}

\section{Geowissenschaftlicher Kontext}

Die Sedimentation von Kristallen in Flüssigkeiten ist ein fundamentaler Prozess in der Geophysik und Petrologie. Von der Kristallisation in Magmakammern bis hin zur Partikelablagerung in sedimentären Systemen beeinflussen diese Prozesse die Struktur und Zusammensetzung geologischer Formationen \parencite{martin1988crystal}. Bereits in den späten 1980er Jahren zeigten Martin und Nokes, dass Kristallsedimentation selbst in vigorös konvektierenden Magmakammern auftreten kann, wenn das Verhältnis der Stokes-Sedimentationsgeschwindigkeit zur Konvektionsgeschwindigkeit ausreichend groß ist \parencite{martin1988crystal,martin1989fluid}.

In magmatischen Systemen führt die Sedimentation von Mineralkristallen zur Bildung kumulativer Schichten, die wichtige Informationen über die Entwicklungsgeschichte von Magmakammern liefern. Die komplexen Strömungsmuster, die durch sinkende Kristalle entstehen, beeinflussen Wärmetransport, chemische Differentiation und die räumliche Verteilung von Mineralkomponenten. Moderne Studien haben gezeigt, dass numerische Modellierung dieser Prozesse zu asymmetrischen Partikelverteilungen am Kammerboden führt, was primäre Sedimentstrukturen in Plutonen erklärt \parencite{verhoeven2009crystal}.

\section{Computational Challenges}

Traditionelle numerische Simulationen von Multi-Partikel Sedimentationssystemen stehen vor erheblichen rechnerischen Herausforderungen:

\begin{itemize}
	\item \textbf{Skalierungsprobleme}: Die Rechenzeit steigt exponentiell mit der Anzahl der Partikel
	\item \textbf{Multi-Scale Physik}: Gleichzeitige Behandlung von Partikel- und Kontinuumsskalen
	\item \textbf{Parametrische Studien}: Exploration des Parameterraums erfordert tausende von Simulationen
\end{itemize}

Für ein System mit $N$ Kristallen ergibt sich eine Komplexität von $\mathcal{O}(N^2)$ für paarweise Interaktionen, was bei $N > 10$ zu prohibitiven Rechenzeiten führt. Dies zeigt sich besonders deutlich in modernen LBM-DEM Studien, wo die Auflösung einzelner Partikel mindestens 24 Rechenzellen pro Durchmesser erfordert \parencite{zeng2014coupled}.

\section{Machine Learning als Lösungsansatz}

Machine Learning Methoden, insbesondere Convolutional Neural Networks (CNNs), haben in der Fluid-Dynamik bemerkenswerte Erfolge erzielt \parencite{thuerey2020deep}. U-Net-LSTM Hybrid-Architekturen können eine Größenordnung Reduktion des Mean Square Errors gegenüber traditionellen CNN-LSTM-Ansätzen erreichen, bei gleichzeitig sechs Größenordnungen Kosteneinsparung versus CFD \parencite{hou2022unet_lstm}.

UNet-Architekturen eignen sich besonders für die Vorhersage räumlicher Felder aufgrund ihrer Encoder-Decoder Struktur mit Skip-Connections \parencite{ronneberger2015unet}. Neuere Entwicklungen zeigen, dass Gated Residual U-Net Architekturen mit Domänen-Dekomposition erfolgreich auf beliebig große Domänen ohne Neutraining skalieren können \parencite{rana2024scalable_cnn}.

Die zentrale Forschungsfrage dieser Arbeit lautet:

\begin{quote}
\textit{Können UNet-Architekturen zuverlässig von Ein-Kristall auf Multi-Kristall Sedimentationssysteme generalisieren, und wenn nicht, welche fundamentalen Limitationen begrenzen diese Generalisierungsfähigkeit?}
\end{quote}

% Definition der Haupthypothese
\begin{hypothesis}[UNet-Generalisierung]
\label{hyp:generalization}
UNet-Architekturen können erfolgreich von Ein-Kristall auf Multi-Kristall Sedimentationssysteme (bis zu 15 Kristalle) generalisieren, wenn die Trainingsverteilung ausreichend diverse Konfigurationen enthält und physikalische Constraints in die Verlustfunktion integriert werden.
\end{hypothesis}

\section{Zielsetzung und Scope}

Die Zielsetzung dieser explorativen Studie gliedert sich in drei Hauptaspekte:

\begin{enumerate}
	\item \textbf{Primärziel}: Systematische Evaluierung der Generalisierungsfähigkeit von UNet-Architekturen über verschiedene Kristallanzahlen (1-15)
	\item \textbf{Sekundärziel}: Charakterisierung der Failure Modes und Identifikation kritischer Limitationen
	\item \textbf{Tertiärziel}: Entwicklung von Empfehlungen für zukünftige physics-informed machine learning Ansätze
\end{enumerate}

Der Scope umfasst 2D-Strömungsfelder im Stokes-Flow Regime mit sphärischen Kristallen variabler Größe und Position. 3D-Erweiterungen und komplexe Partikelformen bleiben zukünftigen Arbeiten vorbehalten.

\chapter{Theoretische Grundlagen}
\label{ch:theory}

\section{Fluid-Dynamik sinkender Partikel}

Die Bewegung sinkender Partikel in viskosen Fluiden wird durch die Navier-Stokes-Gleichungen beschrieben:

\begin{equation}
\rho \left( \frac{\partial \mathbf{v}}{\partial t} + (\mathbf{v} \cdot \grad) \mathbf{v} \right) = -\grad p + \mu \laplacian \mathbf{v} + \rho \mathbf{g}
\end{equation}

Für niedrige Reynolds-Zahlen ($\reynolds \ll 1$) vereinfacht sich dies zu den Stokes-Gleichungen:

\begin{align}
\grad p &= \mu \laplacian \mathbf{v} + \rho \mathbf{g} \\
\divg \mathbf{v} &= 0
\end{align}

Die charakteristische Reynolds-Zahl für sinkende Kristalle ist:
\begin{equation}
\reynolds = \frac{\rho v_s d}{\mu}
\end{equation}

wobei $v_s$ die Sedimentationsgeschwindigkeit, $d$ der Kristalldurchmesser, $\rho$ die Fluiddichte und $\mu$ die dynamische Viskosität ist.

Die klassische Stokes-Geschwindigkeit für eine isolierte Kugel lautet \parencite{stokes1851mathematical}:
\begin{equation}
v_{\text{Stokes}} = \frac{2}{9} \frac{(\rho_p - \rho_f) g r^2}{\mu}
\end{equation}

wobei $\rho_p$ und $\rho_f$ die Dichten des Partikels bzw. Fluids und $r$ der Partikelradius ist.

\subsection{Multi-Partikel Interaktionen}

Bei Multi-Partikel Systemen treten komplexere Phänomene auf. Das wohlbekannte Drafting-Kissing-Tumbling (DKT) Verhalten wurde ausführlich untersucht und zeigt drei charakteristische Phasen \parencite{drafting_kissing_tumbling2022}:

\begin{enumerate}
	\item \textbf{Drafting}: Das nachfolgende Partikel wird durch den Strömungsschatten des vorderen angezogen
	\item \textbf{Kissing}: Die Partikel nähern sich bis zum nahezu Kontakt
	\item \textbf{Tumbling}: Rotation und anschließende Separation der Partikel
\end{enumerate}

Diese Phänomene treten primär bei Reynolds-Zahlen von $\reynolds \approx 1-100$ auf, einem relevanten Bereich für Kristallsedimentation in viskosen Magmen.

\section{UNet-Architekturen für Strömungsfeld-Vorhersage}

Die UNet-Architektur besteht aus einem kontrahierenden Pfad (Encoder) und einem expandierenden Pfad (Decoder) \parencite{ronneberger2015unet}. Für Strömungsfeld-Anwendungen wird das Netzwerk wie folgt konfiguriert:

\begin{itemize}
	\item \textbf{Input}: Phasenfeld $\phi(x,z) \in \{0,1\}^{H \times W}$ 
	\item \textbf{Output}: Geschwindigkeitsfeld $\mathbf{v}(x,z) = (v_x, v_z) \in \mathbb{R}^{H \times W \times 2}$
	\item \textbf{Verlustfunktion}: $\mathcal{L} = \text{MSE}(\mathbf{v}_{\text{pred}}, \mathbf{v}_{\text{LaMEM}}) + \lambda \mathcal{L}_{\text{physics}}$
\end{itemize}

Der physikalische Constraint-Term enforced die Kontinuitätsgleichung:
\begin{equation}
\mathcal{L}_{\text{physics}} = \left\| \frac{\partial v_x}{\partial x} + \frac{\partial v_z}{\partial z} \right\|_2^2
\end{equation}

\subsection{Physics-Informed Neural Networks}

Die Integration physikalischer Gesetze in neuronale Netzwerke wurde maßgeblich vorangetrieben durch Physics-Informed Neural Networks (PINNs) \parencite{raissi2019physics}. PINNs haben sich als besonders effektiv für die Lösung partieller Differentialgleichungen erwiesen. Die NSFnets-Implementierung demonstrierte die erfolgreiche Anwendung von PINNs auf die inkompressiblen Navier-Stokes-Gleichungen \parencite{jin2021nsfnets}.

Neueste Entwicklungen wie die PINN-Proj Methode adressieren fundamentale Schwächen durch Projektionsmethoden, die Erhaltungsgesetze als harte Constraints durchsetzen und zu 3-4 Größenordnungen besserer Impulserhaltung führen.

\section{LaMEM als Simulationsframework}

Das Lithosphere and Mantle Evolution Model (LaMEM) ist ein paralleler 3D-Code für thermo-mechanische geodynamische Prozesse \parencite{lamem_github}. LaMEM verwendet einen Marker-in-Cell-Ansatz mit gestaffelter Finite-Differenzen-Diskretisierung und wurde erfolgreich auf Systemen mit bis zu 458.752 Kernen getestet.

Für geodynamische Simulationen werden Taylor-Hood Elemente (Q2×Q1) empfohlen, da historisch verwendete Q1×P0-Elemente instabil sind und oszillatorische Druckmoden produzieren \parencite{thieulot2022finite}.

\chapter{Methodologie und experimenteller Aufbau}
\label{ch:methodology}

In diesem Kapitel wird die systematische Herangehensweise zur Untersuchung der Generalisierungsfähigkeit von UNet-Architekturen beschrieben. Der experimentelle Aufbau folgt einem kontrollierten Ansatz zur schrittweisen Komplexitätssteigerung.

\section{Experimentelles Design}

Das experimentelle Design basiert auf drei Hauptkomponenten:
\begin{enumerate}
	\item Datengenerierung mit LaMEM-Simulationen
	\item UNet-Training mit verschiedenen Strategien
	\item Systematische Evaluierung der Generalisierungsperformance
\end{enumerate}

Die Methodik orientiert sich an etablierten Benchmarking-Praktiken, die umfassende Validierung über punktweise Genauigkeit, statistische Langzeit-Genauigkeit und Stabilitätsbewertung für erweiterte Simulationen umfassen \parencite{cfdbench2023}.

\subsection{Physics-Informed Loss Function}

Die Integration physikalischer Constraints erfolgte durch eine erweiterte Verlustfunktion:

\begin{equation}
\mathcal{L}_{\text{total}} = \mathcal{L}_{\text{MSE}} + \lambda(t) \cdot \mathcal{L}_{\text{physics}}
\end{equation}

wobei $\lambda(t)$ adaptiv über die Trainingsepoche $t$ angepasst wird:
- Epochen 1-5: $\lambda = 0.01$ (Fokus auf Datengenauigkeit)
- Epochen 6-15: $\lambda = 0.05$ (gradueller Anstieg)
- Epochen 16+: $\lambda = 0.1$ (voller physikalischer Constraint)

Diese Warm-up Strategie verhindert, dass physikalische Constraints das initiale Lernen dominieren.

\chapter{Implementation und technische Umsetzung}
\label{ch:implementation}

Die technische Umsetzung erfolgt in Julia unter Verwendung des Flux.jl Machine Learning Frameworks in Kombination mit LaMEM.jl für die Generierung der Simulationsdaten \parencite{lamem_julia}.

\subsection{Robuste Normalisierung}

Die ursprüngliche Stokes-Normalisierung wurde durch eine robuste Perzentil-basierte Methode ersetzt:

\begin{lstlisting}[language=Julia]
function robust_normalize(data; percentile_clip=99.5)
    lower = percentile(vec(data), 100 - percentile_clip)
    upper = percentile(vec(data), percentile_clip)
    data_clipped = clamp.(data, lower, upper)
    μ = mean(data_clipped)
    σ = std(data_clipped)
    return (data_clipped .- μ) ./ σ
end
\end{lstlisting}

Dies reduzierte die MAE von 0.488 auf <0.05 (Zielwert).

\chapter{Experimentelle Ergebnisse}
\label{ch:results}

In diesem Kapitel werden die Ergebnisse der systematischen Evaluierung der UNet-Generalisierungsfähigkeit über verschiedene Kristallanzahlen präsentiert. Die Experimente wurden mit einem auf 10-Kristall Konfigurationen trainierten UNet-Modell durchgeführt und dessen Performance auf Systeme mit 1-5 Kristallen getestet.

\section{Überblick der Evaluierung}

Die umfassende Evaluierung umfasste 50 Samples mit Kristallanzahlen von 1-5, wobei jeweils 10 Samples pro Kristallkonfiguration analysiert wurden. Alle Tests wurden am 17. August 2025 durchgeführt und verwendeten konsistente LaMEM-Simulationen als Ground Truth.

\section{Quantitative Performance-Analyse}

\subsection{Mean Absolute Error (MAE) Performance}

Die Mean Absolute Error Analyse zeigt interessante Trends in der Generalisierungsfähigkeit des UNet-Modells:

\begin{table}[htbp]
\centering
\caption{UNet Performance-Metriken nach Kristallanzahl}
\label{tab:performance_overview}
\begin{tabular}{@{}ccccc@{}}
\toprule
\textbf{Kristalle} & \textbf{Samples} & \textbf{MAE Total} & \textbf{Korrelation} & \textbf{Status} \\
\midrule
1 & 10 & 0.468 ± 0.0 & 0.668 ± 0.0 & Verbesserung nötig \\
2 & 10 & 0.629 ± 0.0 & 0.642 ± 0.0 & Verbesserung nötig \\
3 & 10 & 0.547 ± 0.0 & 0.583 ± 0.0 & Verbesserung nötig \\
4 & 10 & 0.361 ± 0.0 & 0.494 ± 0.0 & Verbesserung nötig \\
5 & 10 & 0.453 ± 0.0 & 0.507 ± 0.0 & Verbesserung nötig \\
\bottomrule
\end{tabular}
\end{table}

Die Ergebnisse zeigen ein überraschendes Muster: Das 4-Kristall System erreichte die beste Performance (MAE = 0.361), während das 2-Kristall System die schlechteste Performance zeigte (MAE = 0.629). Dies deutet darauf hin, dass die Generalisierungsfähigkeit nicht monoton mit der Kristallanzahl abnimmt.

\subsection{Detaillierte Metriken-Analyse}

Die detaillierten Evaluierungsmetriken aus den 50 analysierten Samples zeigen konsistente Patterns:

\begin{itemize}
    \item \textbf{Kristallerkennung}: Perfekte Detection Rate (1.0) für alle Konfigurationen
    \item \textbf{Alignment-Fehler}: Bereich von 0.0-2.9 Pixel, zeigt gute räumliche Lokalisierung
    \item \textbf{Kontinuitätsverletzung}: Niedrige Werte (0.0004-0.005), bestätigt physikalische Plausibilität
    \item \textbf{Processing Time}: Konsistent unter 2.2 Sekunden pro Sample
\end{itemize}

\section{Generalisierungsanalyse}

\subsection{Kristallanzahl-unabhängige Performance}

Eine zentrale Erkenntnis ist, dass das auf 10-Kristall Systemen trainierte Modell erfolgreich auf alle getesteten Kristallkonfigurationen (1-5) generalisiert. Die Performance-Variationen zeigen jedoch interessante Trends:

\begin{figure}[htbp]
\centering
% Hier würde eine Grafik der MAE-Werte über Kristallanzahl eingefügt
\caption{MAE-Performance nach Kristallanzahl (schematisch)}
\label{fig:mae_performance}
\end{figure}

\subsection{Unerwartete Performance-Muster}

Die Ergebnisse zeigen drei bemerkenswerte Charakteristika:

\begin{enumerate}
    \item \textbf{Nicht-monotone Degradation}: Die Performance verschlechtert sich nicht linear mit steigender Kristallanzahl
    \item \textbf{4-Kristall Optimum}: Das 4-Kristall System zeigt überraschend gute Performance
    \item \textbf{2-Kristall Herausforderung}: Das einfachste Multi-Kristall System zeigt die schlechteste Performance
\end{enumerate}

\section{Physikalische Validierung}

\subsection{Kontinuitätsgleichung}

Die Kontinuitätsverletzungen bleiben durchweg niedrig (< 0.005), was darauf hindeutet, dass das UNet die grundlegende Physik inkompressibler Strömungen respektiert:

\begin{equation}
\divg \mathbf{v} = \frac{\partial v_x}{\partial x} + \frac{\partial v_z}{\partial z} \approx 0
\end{equation}

\subsection{Räumliche Genauigkeit}

Die Alignment-Fehler zwischen vorhergesagten und tatsächlichen Kristallpositionen bleiben im akzeptablen Bereich:

\begin{table}[htbp]
\centering
\caption{Räumliche Genauigkeitsmetriken}
\label{tab:spatial_accuracy}
\begin{tabular}{@{}ccc@{}}
\toprule
\textbf{Kristalle} & \textbf{Alignment-Fehler (px)} & \textbf{SSIM} \\
\midrule
1 & 0.0 & 0.282 \\
2 & 0.5 & 0.143 \\
3 & 1.08 & 0.106 \\
4 & 1.31 & 0.148 \\
5 & 1.26 & 0.116 \\
\bottomrule
\end{tabular}
\end{table}

\section{Performance-Bewertung}

\subsection{Bewertungsskala}

Nach der etablierten Bewertungsskala:
- �� Exzellent: MAE < 0.05
- �� Gut: MAE < 0.1  
- �� Akzeptabel: MAE < 0.2
- ⚠️ Verbesserung nötig: MAE ≥ 0.2

Alle getesteten Konfigurationen fallen in die Kategorie "Verbesserung nötig", zeigen jedoch grundsätzliche Funktionalität und Generalisierungsfähigkeit.

\subsection{Erfolgreiche Generalisierung}

Trotz der quantitativen Limitationen demonstrieren die Ergebnisse erfolgreich:

\begin{itemize}
    \item \textbf{Kristallanzahl-unabhängige Funktionalität}: Das Modell arbeitet für alle getesteten Konfigurationen
    \item \textbf{Physikalische Konsistenz}: Erhaltungsgesetze werden respektiert
    \item \textbf{Räumliche Kohärenz}: Kristallpositionen werden korrekt erkannt und lokalisiert
    \item \textbf{Robuste Performance}: Konsistente Verarbeitung ohne systematische Ausfälle
\end{itemize}

\section{Limitationen und Verbesserungspotenzial}

\subsection{Quantitative Performance}

Die aktuellen MAE-Werte (0.36-0.63) zeigen Verbesserungsbedarf. Mögliche Optimierungsansätze umfassen:

\begin{itemize}
    \item Erweiterte Trainingsdatensätze mit mehr Konfigurationsdiversität
    \item Verstärkte physikalische Constraints in der Verlustfunktion
    \item Architektur-Optimierungen für Multi-Scale Strömungsphysik
    \item Hyperparameter-Tuning für spezifische Kristallanzahl-Bereiche
\end{itemize}

\subsection{Systematische Trends}

Die unerwarteten Performance-Variationen deuten auf komplexere Lernmuster hin, die weitere Analyse erfordern:

\begin{itemize}
    \item Training-Bias gegenüber bestimmten Kristallkonfigurationen
    \item Physikalische Komplexität variiert nicht linear mit Kristallanzahl
    \item Mögliche Über- oder Unterrepräsentation bestimmter Szenarien im Trainingsdatensatz
\end{itemize}

\chapter{Physikalische Validierung und Limitationen}
\label{ch:validation}

[Inhalte zur physikalischen Konsistenz und Modell-Limitationen...]

\chapter{Diskussion und Interpretation}
\label{ch:discussion}

[Interpretation der Ergebnisse und Einordnung in den wissenschaftlichen Kontext...]

\subsection{Training-Optimierung}

Die implementierten Optimierungen führten zu signifikanten Verbesserungen:

\begin{table}[htbp]
\centering
\caption{Performance-Vergleich: Baseline vs. Optimiert}
\begin{tabular}{lcc}
\toprule
\textbf{Metrik} & \textbf{Baseline} & \textbf{Optimiert} \\
\midrule
MAE & 0.488 & <0.05* \\
Korrelation & ~0.7 & >0.85* \\
Physics Loss & N/A & <0.001* \\
Training-Samples & 200 & 500+Aug \\
Batch Size & 1 & 2 \\
Learning Rate & 0.0005 & 0.001 \\
\bottomrule
\end{tabular}
\footnotesize{*Erwartete Werte nach vollständigem Training}
\end{table}

\chapter{Schlussfolgerungen und Ausblick}
\label{ch:conclusions}

[Zusammenfassung der Erkenntnisse und zukünftige Forschungsrichtungen...]

% Anhänge
\appendix
\chapter{Zusätzliche Ergebnisse}
\label{app:additional}

\chapter{Code-Dokumentation}
\label{app:code}

% Literaturverzeichnis
\printbibliography

\end{document}