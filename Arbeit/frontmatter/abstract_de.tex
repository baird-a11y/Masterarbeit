\begin{abstract}
Diese Masterarbeit untersucht die Generalisierungsfähigkeit von UNet-Architekturen zur Vorhersage von Strömungsfeldern in Multi-Kristall Sedimentationssystemen. Während traditionelle computational fluid dynamics (CFD) Methoden für komplexe Multi-Partikel Systeme rechenzeitintensiv sind, bieten machine learning Ansätze das Potenzial für erhebliche Beschleunigungen. Die vorliegende explorative Studie evaluiert systematisch, ob UNet-Netzwerke, die auf Ein-Kristall Konfigurationen trainiert wurden, erfolgreich auf Systeme mit 2-15 Kristallen generalisieren können. 

Mithilfe von LaMEM-Simulationen als Ground Truth werden kontrollierte Experimente zur Komplexitätssteigerung durchgeführt. Die Ergebnisse zeigen [Platzhalter für tatsächliche Resultate], wodurch sowohl die Möglichkeiten als auch die fundamentalen Limitationen von UNet-Generalisierung in Multi-Partikel Strömungssystemen charakterisiert werden. Diese Arbeit leistet einen methodischen Beitrag zur physics-informed machine learning Forschung und bietet praktische Erkenntnisse für geowissenschaftliche Modellierungsanwendungen.
\end{abstract}