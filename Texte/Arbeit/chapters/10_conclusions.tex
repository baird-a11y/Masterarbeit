% ----------------------------------------------------------------------------
% Chapter 10: Conclusion
% ----------------------------------------------------------------------------
%
% ============================================================
% QUELLENVORSCHLÄGE FÜR DIESES KAPITEL
% ============================================================
% Zusammenfassung der Beiträge (verweist auf Literatur, die du übertroffen/ergänzt hast):
%   \parencite{thuerey_deep_2020}            -- Bisheriger Stand: U-Net für variable Geometrien
%   \parencite{li_fourier_2021}              -- Bisheriger Stand: FNO für PDE-Surrogate
%   \parencite{lu_comprehensive_2022}        -- Fairer Architekturvergleich als Referenzpunkt
%   \parencite{morimoto_generalization_2022} -- Generalisierungsanalyse als methodischer Vorläufer
%
% Ausblick / Zukünftige Arbeiten:
%   \parencite{wen_u-fnoenhanced_2022}       -- U-FNO: Hybridarchitektur als nächster Schritt
%   \parencite{qin_toward_2024}              -- Verbesserung Spektralbias (adaptive Modes)
%   \parencite{sanchez-gonzalez_learning_nodate} -- GNN für größere Partikelzahlen
%   \parencite{zhu_bayesian_2018}            -- Unsicherheitsquantifizierung
%   \parencite{hu_applying_2023}             -- PINN-Erweiterung für physikkonsistente Vorhersagen
%   \parencite{koehler_apebench_nodate}      -- Autoregressive Erweiterungen (zeitabhängige Settl.)
%   \parencite{li_fourier_nodate}            -- Geo-FNO für irreguläre Kristallgeometrien
%   \parencite{patocka_settling_2020}        -- Breitere geodynamische Anwendbarkeit
% ============================================================
%
\chapter{Conclusion and Outlook}
\label{ch:conclusions}

Hier soll eine Zusammenfassung der wichtigsten Ergebnisse meiner Arbeit gegeben werden, sowie eine Diskussion darüber, welche Implikationen diese Ergebnisse für die Praxis und die Forschung haben. Ich werde auch einen Ausblick auf mögliche zukünftige Arbeiten geben, die auf den Ergebnissen meiner Arbeit aufbauen könnten. Es könnte auch eine Reflexion darüber geben, welche Herausforderungen ich bei der Durchführung meiner Experimente und der Analyse meiner Ergebnisse hatte und wie ich diese Herausforderungen bewältigt habe. Schließlich werde ich die wichtigsten Erkenntnisse meiner Arbeit noch einmal zusammenfassen und betonen, warum sie wichtig sind und welche Auswirkungen sie haben könnten.
