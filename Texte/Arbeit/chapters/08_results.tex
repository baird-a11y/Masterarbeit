% ----------------------------------------------------------------------------
% Chapter 8: Results
% ----------------------------------------------------------------------------
%
% ============================================================
% QUELLENVORSCHLÄGE FÜR DIESES KAPITEL
% ============================================================
% Einordnung der Ergebnisse (U-Net vs. FNO Vergleich):
%   \parencite{lu_comprehensive_2022}        -- Fairer Vergleich neural operators auf PDE-Benchmarks
%   \parencite{wen_u-fnoenhanced_2022}       -- U-FNO: Hybridarchitektur als Vergleichspunkt
%   \parencite{qin_toward_2024}              -- Spektralbias des FNO nahe scharfer Grenzflächen
%   \parencite{raonic_convolutional_2023}    -- Convolutional Neural Operators als Referenz
%
% Generalisierung über Kristallzahlen:
%   \parencite{morimoto_generalization_2022} -- Generalisierungstechniken für NN in Strömungen
%   \parencite{rana_scalable_2024}           -- Skalierbarkeit CNN-Surrogate auf mehr Hindernisse
%   \parencite{thuerey_deep_2020}            -- Generalisierung über Geometrien (U-Net)
%
% Fehleranalyse / Spatial Error Patterns:
%   \parencite{takamoto_pdebench_nodate}     -- Benchmark-Metriken zum Vergleich
%   \parencite{ribeiro_deepcfd_nodate}       -- Fehleranalyse nach geometrischen Regionen
%   \parencite{chen_u-net_nodate}            -- Räumliche Fehlerverteilung bei U-Net-Surrogaten
%
% Einbettung in Geowissenschaften:
%   \parencite{martin_crystal_1988}          -- Erwartete Fehler in Interaktionsregimen
%   \parencite{verhoeven_numerical_2009}     -- Numerische Referenzlösungen für Kristallsedimentation
%   \parencite{patocka_settling_2020}        -- Weitere numerische Referenz für Sedimentation
% ============================================================
%
\chapter{Results}
\label{ch:results}

This chapter presents the results of the experiments described in Chapter~\ref{ch:experiments}.
All experiments target the prediction of the stream function~$\psi$ for Stokes flow around
settling crystals. The evaluation covers training convergence, quantitative error metrics on
a held-out evaluation set, and qualitative inspection of predicted flow fields.
Results are organized by experiment; within each experiment, FNO and U-Net are presented
separately and then compared.
The learning rate values for each configuration will be specified once the final experimental
setups are confirmed.

% ============================================================================
\section{Experiment~1: Single-Crystal Baseline}
\label{sec:results:exp1}
% ============================================================================

Experiment~1 establishes a per-architecture baseline on the simplest configuration:
single-crystal flows ($n = 1$). The sole free parameter across the four configurations
is the learning rate; all other hyperparameters are fixed at the architecture defaults
(see Section~\ref{sec:experiments:setup}).
The four configurations are referred to as \emph{Exp-1.1} through \emph{Exp-1.4},
with learning rates \TODO{lr values werden ergänzt}.

% ----------------------------------------------------------------------------
\subsection{FNO – Training Behavior}
\label{ssec:results:exp1:fno:training}
% ----------------------------------------------------------------------------

Figure~\ref{fig:fno_training_exp1} shows the training and validation loss curves for all four
FNO configurations over 100~epochs. All runs converge monotonically in training loss; the
validation curves exhibit mild fluctuations but generally track the training loss, indicating
no severe overfitting.

\begin{figure}[htbp]
    \centering
    \begin{subfigure}[b]{0.48\textwidth}
        \includegraphics[width=\textwidth]{../../Ergebnisse/FNO_Ergebnisse/One_Crystal/eval_output_one/plots/training_history.png}
        \caption{Exp-1.1 (lr $= $ \TODO{lr\_1})}
        \label{fig:fno_train_one}
    \end{subfigure}
    \hfill
    \begin{subfigure}[b]{0.48\textwidth}
        \includegraphics[width=\textwidth]{../../Ergebnisse/FNO_Ergebnisse/One_Crystal/eval_output_two/plots/training_history.png}
        \caption{Exp-1.2 (lr $= $ \TODO{lr\_2})}
        \label{fig:fno_train_two}
    \end{subfigure}
    \vspace{0.5em}
    \begin{subfigure}[b]{0.48\textwidth}
        \includegraphics[width=\textwidth]{../../Ergebnisse/FNO_Ergebnisse/One_Crystal/eval_output_three/plots/training_history.png}
        \caption{Exp-1.3 (lr $= $ \TODO{lr\_3})}
        \label{fig:fno_train_three}
    \end{subfigure}
    \hfill
    \begin{subfigure}[b]{0.48\textwidth}
        \includegraphics[width=\textwidth]{../../Ergebnisse/FNO_Ergebnisse/One_Crystal/eval_output_four/plots/training_history.png}
        \caption{Exp-1.4 (lr $= $ \TODO{lr\_4})}
        \label{fig:fno_train_four}
    \end{subfigure}
    \caption{FNO training and validation loss curves for the four learning-rate configurations
             of Experiment~1 (single crystal, 100~epochs).
             Training MSE (solid) and validation relative $L_2$ error (dashed) per epoch.}
    \label{fig:fno_training_exp1}
\end{figure}

% ----------------------------------------------------------------------------
\subsection{FNO – Evaluation Metrics}
\label{ssec:results:exp1:fno:metrics}
% ----------------------------------------------------------------------------

After training, each FNO model is evaluated on a fixed set of ten single-crystal samples
from the held-out evaluation set. Table~\ref{tab:fno_metrics_exp1} summarizes the aggregated
metrics. The divergence RMS is included as a physical sanity check; since the stream-function
formulation guarantees $\nabla \cdot \mathbf{u} = 0$ by construction, its values are
numerically negligible ($\sim 10^{-26}$) for all configurations.

\begin{table}[htbp]
\centering
\caption{FNO evaluation metrics for Experiment~1 (single crystal, $n=1$, 10~samples).
         $\bar{\varepsilon}_\psi$: mean relative $L_2$ error on $\psi$;
         $\bar{\varepsilon}_v$: mean relative $L_2$ error on derived velocity;
         $\bar{e}_{\max}$: mean maximum absolute error on $\psi$;
         $\overline{\text{MSE}}_\psi$: mean squared error on $\psi$.
         Standard deviations over the 10 samples in parentheses.}
\label{tab:fno_metrics_exp1}
\begin{tabular}{lcccc}
\toprule
\textbf{Config} & $\bar{\varepsilon}_\psi$ (\%) & $\bar{\varepsilon}_v$ (\%) & $\bar{e}_{\max}$ & $\overline{\text{MSE}}_\psi$ \\
\midrule
Exp-1.1 (lr $=$ \TODO{})
    & $1.85 \pm 0.65$ & $3.84 \pm 0.58$ & $0.360 \pm 0.168$ & $0.0121 \pm 0.0121$ \\
Exp-1.2 (lr $=$ \TODO{})
    & $3.23 \pm 1.52$ & $5.34 \pm 1.13$ & $0.672 \pm 0.412$ & $0.0487 \pm 0.0526$ \\
Exp-1.3 (lr $=$ \TODO{})
    & $5.72 \pm 2.56$ & $8.48 \pm 1.72$ & $0.934 \pm 0.426$ & $0.131  \pm 0.142$  \\
Exp-1.4 (lr $=$ \TODO{})
    & $2.58 \pm 0.91$ & $5.10 \pm 0.59$ & $0.548 \pm 0.239$ & $0.0260 \pm 0.0267$ \\
\bottomrule
\end{tabular}
\end{table}

Exp-1.1 yields the lowest errors across all metrics, achieving a mean relative $L_2$ error
of $1.85\%$ on~$\psi$ and $3.84\%$ on the derived velocity field.
Exp-1.4 is the second-best configuration ($2.58\%$ / $5.10\%$), while the higher learning
rates of Exp-1.2 and Exp-1.3 produce noticeably larger errors.
This indicates that the FNO benefits from a moderate-to-small learning rate when trained
with the Adam optimizer and gradient clipping at norm~1.0.

Figure~\ref{fig:fno_metrics_vs_crystals_exp1} shows the per-configuration metric plots
as generated by the evaluation pipeline.

\begin{figure}[htbp]
    \centering
    \begin{subfigure}[b]{0.48\textwidth}
        \includegraphics[width=\textwidth]{../../Ergebnisse/FNO_Ergebnisse/One_Crystal/eval_output_one/plots/metrics_vs_crystals.png}
        \caption{Exp-1.1 (lr $= $ \TODO{lr\_1})}
    \end{subfigure}
    \hfill
    \begin{subfigure}[b]{0.48\textwidth}
        \includegraphics[width=\textwidth]{../../Ergebnisse/FNO_Ergebnisse/One_Crystal/eval_output_two/plots/metrics_vs_crystals.png}
        \caption{Exp-1.2 (lr $= $ \TODO{lr\_2})}
    \end{subfigure}
    \vspace{0.5em}
    \begin{subfigure}[b]{0.48\textwidth}
        \includegraphics[width=\textwidth]{../../Ergebnisse/FNO_Ergebnisse/One_Crystal/eval_output_three/plots/metrics_vs_crystals.png}
        \caption{Exp-1.3 (lr $= $ \TODO{lr\_3})}
    \end{subfigure}
    \hfill
    \begin{subfigure}[b]{0.48\textwidth}
        \includegraphics[width=\textwidth]{../../Ergebnisse/FNO_Ergebnisse/One_Crystal/eval_output_four/plots/metrics_vs_crystals.png}
        \caption{Exp-1.4 (lr $= $ \TODO{lr\_4})}
    \end{subfigure}
    \caption{FNO evaluation metric summaries for Experiment~1. Each panel shows the mean
             relative $L_2$ errors on $\psi$ and derived velocity, stratified by crystal count
             (here only $n=1$).}
    \label{fig:fno_metrics_vs_crystals_exp1}
\end{figure}

% ----------------------------------------------------------------------------
\subsection{FNO – Qualitative Results}
\label{ssec:results:exp1:fno:qualitative}
% ----------------------------------------------------------------------------

Figure~\ref{fig:fno_gallery_exp1} shows representative predictions from Exp-1.1 (the
best-performing FNO configuration). The stream function field~$\psi$ and the derived
velocity magnitude $|\mathbf{u}|$ are displayed alongside the reference solution.

\begin{figure}[htbp]
    \centering
    \begin{subfigure}[b]{0.48\textwidth}
        \includegraphics[width=\textwidth]{../../Ergebnisse/FNO_Ergebnisse/One_Crystal/eval_output_one/plots/gallery/n01/004_rel0.0172_psi.png}
        \caption{Stream function $\psi$ (rel.\ error $= 1.72\%$)}
        \label{fig:fno_gallery_psi}
    \end{subfigure}
    \hfill
    \begin{subfigure}[b]{0.48\textwidth}
        \includegraphics[width=\textwidth]{../../Ergebnisse/FNO_Ergebnisse/One_Crystal/eval_output_one/plots/gallery/n01/004_rel0.0172_vel.png}
        \caption{Velocity field $|\mathbf{u}|$ (same sample)}
        \label{fig:fno_gallery_vel}
    \end{subfigure}
    \caption{FNO Exp-1.1: predicted vs.\ reference stream function (left) and velocity
             magnitude (right) for a representative single-crystal sample
             (relative $L_2$ error $= 1.72\%$ on $\psi$).}
    \label{fig:fno_gallery_exp1}
\end{figure}

% ----------------------------------------------------------------------------
\subsection{U-Net – Training Behavior}
\label{ssec:results:exp1:unet:training}
% ----------------------------------------------------------------------------

Figure~\ref{fig:unet_training_exp1} shows the training and validation loss curves for the
four U-Net configurations. All four runs were recorded over 50~epochs; the full 300-epoch
training planned in Section~\ref{sec:experiments:setup} was not yet completed at the time
of writing. The results presented here are therefore preliminary.

The curves of Exp-1.1 and Exp-1.2 exhibit pronounced oscillations in validation loss
throughout all 50~epochs, suggesting that the respective learning rates are too large for
stable U-Net convergence. Exp-1.3 shows the smoothest and most rapid descent, reaching the
lowest validation loss by epoch~50. Exp-1.4 falls between the two regimes with moderate
oscillation and intermediate final loss. It should be noted that Exp-1.2 was restarted
after an initial three-epoch run; the continued run is shown in the figure.

\begin{figure}[htbp]
    \centering
    \begin{subfigure}[b]{0.48\textwidth}
        \includegraphics[width=\textwidth]{../../Ergebnisse/UNET_Ergebnisse/One_Crystal/exp_one/training_history.png}
        \caption{Exp-1.1 (lr $= $ \TODO{lr\_1})}
        \label{fig:unet_train_one}
    \end{subfigure}
    \hfill
    \begin{subfigure}[b]{0.48\textwidth}
        \includegraphics[width=\textwidth]{../../Ergebnisse/UNET_Ergebnisse/One_Crystal/exp_two/training_history.png}
        \caption{Exp-1.2 (lr $= $ \TODO{lr\_2})}
        \label{fig:unet_train_two}
    \end{subfigure}
    \vspace{0.5em}
    \begin{subfigure}[b]{0.48\textwidth}
        \includegraphics[width=\textwidth]{../../Ergebnisse/UNET_Ergebnisse/One_Crystal/exp_three/training_history.png}
        \caption{Exp-1.3 (lr $= $ \TODO{lr\_3})}
        \label{fig:unet_train_three}
    \end{subfigure}
    \hfill
    \begin{subfigure}[b]{0.48\textwidth}
        \includegraphics[width=\textwidth]{../../Ergebnisse/UNET_Ergebnisse/One_Crystal/exp_four/training_history.png}
        \caption{Exp-1.4 (lr $= $ \TODO{lr\_4})}
        \label{fig:unet_train_four}
    \end{subfigure}
    \caption{U-Net training and validation loss curves for the four learning-rate
             configurations of Experiment~1 (single crystal, 50~epochs shown, preliminary).
             Training MSE (solid) and validation relative $L_2$ error (dashed) per epoch.}
    \label{fig:unet_training_exp1}
\end{figure}

% ----------------------------------------------------------------------------
\subsection{U-Net – Metrics at Epoch~50}
\label{ssec:results:exp1:unet:metrics}
% ----------------------------------------------------------------------------

Table~\ref{tab:unet_metrics_exp1} reports the validation MSE and relative $L_2$ error at the
final recorded epoch~(50) for each U-Net configuration. Because training was halted before
convergence, these values are preliminary estimates rather than final model quality.
Updated results will replace these once the full 300-epoch runs are completed.

\begin{table}[htbp]
\centering
\caption{U-Net validation metrics at epoch~50 for Experiment~1 (single crystal, $n=1$,
         preliminary). Values are taken from the last epoch of the training log.}
\label{tab:unet_metrics_exp1}
\begin{tabular}{lcc}
\toprule
\textbf{Config} & $\varepsilon_\psi^{\mathrm{val}}$ (rel.\ $L_2$, \%) & $\mathrm{MSE}_\psi^{\mathrm{val}}$ \\
\midrule
Exp-1.1 (lr $=$ \TODO{}) & $11.2$ & $0.283$ \\
Exp-1.2 (lr $=$ \TODO{}) & $11.2$ & $0.311$ \\
Exp-1.3 (lr $=$ \TODO{}) & $\phantom{0}5.0$  & $0.064$ \\
Exp-1.4 (lr $=$ \TODO{}) & $\phantom{0}6.7$  & $0.117$ \\
\bottomrule
\end{tabular}
\end{table}

Even at this preliminary stage, Exp-1.3 achieves a validation relative $L_2$ error of
approximately $5\%$. Exp-1.4 reaches $6.7\%$, while Exp-1.1 and Exp-1.2 remain around
$11\%$, consistent with their noisier convergence behavior.

% ----------------------------------------------------------------------------
\subsection{U-Net – Qualitative Results}
\label{ssec:results:exp1:unet:qualitative}
% ----------------------------------------------------------------------------

Figure~\ref{fig:unet_gallery_exp1} shows representative evaluation plots for a U-Net
configuration. The overall flow structure is reproduced qualitatively, but local errors
near the crystal boundary appear more pronounced than in the corresponding FNO predictions.

\begin{figure}[htbp]
    \centering
    \begin{subfigure}[b]{0.48\textwidth}
        \includegraphics[width=\textwidth]{../../Ergebnisse/UNET_Ergebnisse/One_Crystal/eval_plots_one/n_01/sample_0001_rel0.1274_psi.png}
        \caption{Exp-1.1: stream function $\psi$ (rel.\ error $= 12.7\%$)}
    \end{subfigure}
    \hfill
    \begin{subfigure}[b]{0.48\textwidth}
        \includegraphics[width=\textwidth]{../../Ergebnisse/UNET_Ergebnisse/One_Crystal/eval_plots_one/n_01/sample_0001_rel0.1274_vel.png}
        \caption{Exp-1.1: velocity $|\mathbf{u}|$ (same sample)}
    \end{subfigure}
    \caption{U-Net Exp-1.1: predicted vs.\ reference stream function (left) and velocity
             magnitude (right) for a representative single-crystal sample
             (relative $L_2$ error $= 12.7\%$ on $\psi$, preliminary 50-epoch model).}
    \label{fig:unet_gallery_exp1}
\end{figure}

% ----------------------------------------------------------------------------
\subsection{Comparison: FNO vs.\ U-Net}
\label{ssec:results:exp1:comparison}
% ----------------------------------------------------------------------------

Table~\ref{tab:comparison_exp1} compares the best result from each architecture in
Experiment~1. The FNO (Exp-1.1) achieves a mean relative $L_2$ error of $1.85\%$ on~$\psi$,
more than $2.5\times$ lower than the best U-Net configuration at its current training stage
(Exp-1.3: $5.0\%$). The velocity field error follows the same pattern ($3.84\%$ vs.\ not
yet available for U-Net on the evaluation set).

It must be emphasized that the U-Net comparison is based on a 50-epoch training run whereas
the full schedule foresees 300~epochs; the gap may therefore be smaller once training is
complete. Nonetheless, the FNO converges to a lower error level in fewer epochs (100 vs.\
50 already evaluated), suggesting a structural advantage of the spectral architecture for
this class of smooth, globally-structured flow fields.

Both models benefit from the stream-function output representation: the divergence RMS of
all FNO predictions is on the order of $10^{-26}$, confirming that incompressibility is
preserved analytically irrespective of prediction accuracy.

\begin{table}[htbp]
\centering
\caption{Best result per architecture in Experiment~1 (single crystal, $n=1$).
         FNO: metrics on the held-out evaluation set (10~samples).
         U-Net: validation metrics at epoch~50 (preliminary).}
\label{tab:comparison_exp1}
\begin{tabular}{llcc}
\toprule
\textbf{Architecture} & \textbf{Config} & $\bar{\varepsilon}_\psi$ (rel.\ $L_2$, \%) & $\bar{\varepsilon}_v$ (rel.\ $L_2$, \%) \\
\midrule
FNO   & Exp-1.1 (lr $=$ \TODO{}) & $1.85$ & $3.84$ \\
U-Net & Exp-1.3 (lr $=$ \TODO{}) & $5.0\phantom{0}$ & --- \\
\bottomrule
\end{tabular}
\end{table}

% ============================================================================
\section{Experiment~2: Multi-Crystal Generalization}
\label{sec:results:exp2}
% ============================================================================

\TODO{Ergebnisse werden nach Abschluss der Experimente ergänzt (vgl.\ Abschnitt~\ref{sec:experiments:configs}).}

% ============================================================================
\section{Experiment~3: Dataset Size Ablation}
\label{sec:results:exp3}
% ============================================================================

\TODO{Ergebnisse werden nach Abschluss der Experimente ergänzt (vgl.\ Abschnitt~\ref{sec:experiments:configs}).}

% ============================================================================
\section{Experiment~4: Architecture Ablation}
\label{sec:results:exp4}
% ============================================================================

\TODO{Ergebnisse werden nach Abschluss der Experimente ergänzt (vgl.\ Abschnitt~\ref{sec:experiments:configs}).}