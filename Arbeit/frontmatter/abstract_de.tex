% ============================================================================
% GERMAN ABSTRACT
% ============================================================================
\begin{abstract}
Diese Masterarbeit untersucht die Generalisierungsfähigkeit von UNet-Architekturen zur Vorhersage von Strömungsfeldern in Multi-Kristall-Sedimentationssystemen. Während traditionelle Computational Fluid Dynamics (CFD) Methoden für komplexe Multi-Partikel-Systeme rechenintensiv sind, bieten Machine Learning Ansätze das Potenzial für erhebliche Beschleunigungen.

Die vorliegende explorative Studie evaluiert systematisch, ob UNet-Netzwerke, die auf einer festen Kristallanzahl (z.B. 10 Kristalle) trainiert wurden, erfolgreich auf beliebige andere Kristallkonfigurationen (1--15 Kristalle) generalisieren können. Unter Verwendung von LaMEM-Simulationen als Ground Truth werden kontrollierte Experimente durchgeführt, bei denen das auf $N$ Kristallen trainierte Modell auf das gesamte Spektrum von 1 bis $N$ Kristallen evaluiert wird. Die Ergebnisse zeigen [Platzhalter für tatsächliche Befunde] und charakterisieren damit sowohl die Möglichkeiten als auch die fundamentalen Limitationen der UNet-Generalisierung über verschiedene Partikelanzahlen hinweg.

Diese Arbeit leistet einen methodischen Beitrag zur physics-informed Machine Learning Forschung und bietet praktische Erkenntnisse für geowissenschaftliche Modellierungsanwendungen.
\end{abstract}
