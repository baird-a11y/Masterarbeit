% ============================================================================
% EIGENE BEFEHLE UND DEFINITIONEN
% ============================================================================

% ---- Mathematische Operatoren ----
%\newcommand{\grad}{\nabla}
%\newcommand{\divg}{\nabla \cdot}
%\newcommand{\curl}{\nabla \times}
%\newcommand{\laplacian}{\nabla^2}

% ---- Spezielle Definitionen für Geowissenschaften ----
\newcommand{\reynolds}{\text{Re}}
\newcommand{\stokes}{\text{St}}
\newcommand{\peclet}{\text{Pe}}

% ---- Platzhalter für offene Stellen ----
\newcommand{\TODO}[1]{\textcolor{red}{\textbf{[TODO: #1]}}}

% ---- Machine Learning / Neural Networks ----
\newcommand{\psihat}{\hat{\psi}}
\newcommand{\fnoblock}{\texttt{FNOBlock}}
\newcommand{\spectconv}{\texttt{SpectralConv2D}}
\newcommand{\jl}[1]{\texttt{#1}}
\newcommand{\modul}[1]{\textbf{\texttt{#1}}}

% ---- Abkürzungen für häufige Begriffe ----
\newcommand{\unet}{UNet}
\newcommand{\lamem}{LaMEM}
\newcommand{\mae}{MAE}
\newcommand{\mse}{MSE}

% ---- Theorem-Umgebungen ----
\theoremstyle{definition}
\newtheorem{definition}{Definition}[chapter]
\newtheorem{hypothesis}{Hypothesis}[chapter]

\theoremstyle{plain}
\newtheorem{theorem}{Theorem}[chapter]
\newtheorem{lemma}[theorem]{Lemma}

% ---- tcolorbox Umgebungen für Hinweise und Warnungen ----
\newtcolorbox{hinweis}[1][]{
  colback=blue!5!white,
  colframe=blue!60!black,
  fonttitle=\bfseries,
  title={Hinweis},
  #1
}

\newtcolorbox{warnung}[1][]{
  colback=red!5!white,
  colframe=red!60!black,
  fonttitle=\bfseries,
  title={Warnung},
  #1
}

