\chapter*{Zusammenfassung}
\addcontentsline{toc}{chapter}{Zusammenfassung}

\begin{otherlanguage}{ngerman}
Kristallsedimentation in viskosen magmatischen Fluiden ist ein fundamentaler
geodynamischer Prozess, der die chemische Differentiation und innere Struktur
von Magmakammern steuert. Die numerisch exakte Simulation von
Mehrkristall-Sedimentation erfordert die gleichzeitige Auflösung mikroskopischer
Grenzschichten um einzelne Kristalle und weitreichender hydrodynamischer
Wechselwirkungen über das gesamte Strömungsgebiet. Dieser ausgeprägt multiskalige
Charakter, kombiniert mit dem nahezu quadratischen Skalierungsverhalten der
Interaktionskosten mit der Partikelzahl, macht systematische Parameterstudien
mit klassischen Solvern rechnerisch kaum durchführbar.

Diese Masterarbeit untersucht, ob datengetriebene Surrogatmodelle
Hochpräzisions-Stokes-Solver für die Strömungsfeldvorhersage in
Mehrkristall-Sedimentationsszenarien ersetzen können. Zwei neuronale
Netzarchitekturen werden auf identischen Daten und Evaluationsmetriken
systematisch verglichen: das U-Net, ein Encoder-Decoder-Modell mit
Skip-Connections zur Extraktion räumlich hierarchischer Merkmale, und der
Fourier Neural Operator (FNO), der im Spektralbereich operiert und pro Schicht
ein globales rezeptives Feld besitzt. Beide Modelle werden darauf trainiert, die
Stromfunktion~$\psi$ aus einer fünfkanaligen geometrischen Eingabekodierung
vorherzusagen, bestehend aus einer Kristallmaske, einem Signed-Distance-Proxy,
dem Abstand zur nächsten Kristallgrenze sowie normierten Raumkoordinaten. Die
Vorhersage von~$\psi$ statt direkter Geschwindigkeitskomponenten erzwingt
Inkompressibilität per Konstruktion und eliminiert so eine häufige Quelle
systematischer Verletzungen physikalischer Nebenbedingungen in rein
datengetriebenen Strömungssurrogaten.

Trainingsdaten werden mit dem geodynamischen Finite-Elemente-Code
\textsc{LaMEM} generiert und liefern stationäre inkompressible Stokes-Lösungen
für Konfigurationen mit $1$ bis $N_{\max}$ zufällig platzierten starren
kreisförmigen Kristallen auf einem festen $256 \times 256$-Gitter. 

Genaue Zahlen muss ich noch einpflegen sobald ich die Versuche gemacht habe. 


Aus~$\psi$ abgeleitete Geschwindigkeitsfelder erweisen sich
konsistent als schwieriger zu reproduzieren als die Stromfunktion selbst, mit
relativen Fehlern, die rund eine Größenordnung größer sind und die
Hochfrequenzverstärkung numerischer Differentiation widerspiegeln.
Fehlerdiagnostik zeigt zudem, dass die Vorhersageschwierigkeit an
Kristallrändern und Domänengrenzen erhöht ist.

Der kontrollierte Vergleich der U-Net- und FNO-Architekturen auf diesem Benchmark
quantifiziert den Zielkonflikt zwischen räumlich lokaler Multiskalenverarbeitung
und globaler Spektraldarstellung für die Surrogatmodellierung elliptischer
Stokes-Strömungen mit mehreren Einschlüssen. Die Ergebnisse liefern praktische
Richtlinien für Eingabekodierung, Ausgaberepräsentation und Architekturwahl in
geophysikalischen Sedimentationsanwendungen.
\end{otherlanguage}
