% ============================================================================
% CHAPTER 1: INTRODUCTION AND RESEARCH MOTIVATION
% ============================================================================

\chapter{Introduction and Research Motivation}
\label{ch:introduction}

\section{Geoscientific Context}

Crystal sedimentation in fluids is a fundamental process in geophysics and petrology. From crystallization in magma chambers to particle deposition in sedimentary systems, these processes influence the structure and composition of geological formations \parencite{martin1988crystal}. As early as the late 1980s, Martin and Nokes demonstrated that crystal sedimentation can occur even in vigorously convecting magma chambers when the ratio of Stokes settling velocity to convection velocity is sufficiently large \parencite{martin1988crystal,martin1989fluid}.

In magmatic systems, the sedimentation of mineral crystals leads to the formation of cumulate layers that provide crucial information about the evolution history of magma chambers. The complex flow patterns generated by settling crystals influence heat transport, chemical differentiation, and the spatial distribution of mineral components. Modern studies have shown that numerical modeling of these processes leads to asymmetric particle distributions at the chamber floor, explaining primary sediment structures in plutons \parencite{verhoeven2009crystal}.

\section{Computational Challenges}

Traditional numerical simulations of multi-particle sedimentation systems face significant computational challenges:

\begin{itemize}
    \item \textbf{Scaling problems}: Computational time increases exponentially with particle number
    \item \textbf{Multi-scale physics}: Simultaneous treatment of particle and continuum scales
    \item \textbf{Parametric studies}: Exploration of parameter space requires thousands of simulations
\end{itemize}

For a system with $N$ crystals, the complexity is $\mathcal{O}(N^2)$ for pairwise interactions, leading to prohibitive computational times for $N > 10$. This is particularly evident in modern LBM-DEM studies, where resolving individual particles requires at least 24 computational cells per diameter \parencite{zeng2014coupled}.

\section{Machine Learning as a Solution Approach}

Machine learning methods, particularly Convolutional Neural Networks (CNNs), have achieved remarkable success in fluid dynamics \parencite{thuerey2020deep}. U-Net-LSTM hybrid architectures can achieve an order-of-magnitude reduction in mean square error compared to traditional CNN-LSTM approaches, while simultaneously providing six orders of magnitude cost savings versus CFD \parencite{hou2022unet_lstm}.

UNet architectures are particularly well-suited for predicting spatial fields due to their encoder-decoder structure with skip connections \parencite{ronneberger2015unet}. Recent developments show that Gated Residual U-Net architectures with domain decomposition can successfully scale to arbitrarily large domains without retraining \parencite{rana2024scalable_cnn}.

The central research question of this thesis is:

\begin{quote}
\textit{Can UNet architectures trained on a fixed number of crystals ($N$) reliably generalize to arbitrary crystal configurations ranging from 1 to $N$ crystals, and if not, what fundamental limitations constrain this generalization capability?}
\end{quote}

% Definition of main hypothesis
\begin{hypothesis}[UNet Generalization Across Crystal Numbers]
\label{hyp:generalization}
UNet architectures trained on a fixed number of crystals ($N$) can successfully generalize to arbitrary crystal configurations ranging from 1 to $N$ crystals, provided the training distribution contains sufficiently diverse spatial arrangements and physical constraints are integrated into the loss function.
\end{hypothesis}

\section{Objectives and Scope}

The objectives of this exploratory study are structured into three main aspects:

\begin{enumerate}
    \item \textbf{Primary objective}: Systematic evaluation of the generalization capability of UNet architectures across different crystal numbers (1-15)
    \item \textbf{Secondary objective}: Characterization of failure modes and identification of critical limitations
    \item \textbf{Tertiary objective}: Development of recommendations for future physics-informed machine learning approaches
\end{enumerate}

The scope encompasses 2D flow fields in the Stokes flow regime with spherical crystals of variable size and position. Extensions to 3D and complex particle shapes are reserved for future work.