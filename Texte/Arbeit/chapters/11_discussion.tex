% ----------------------------------------------------------------------------
% Chapter 11: Discussion
% ----------------------------------------------------------------------------
%
% ============================================================
% QUELLENVORSCHLÄGE FÜR DIESES KAPITEL
% ============================================================
% Architekturvergleich U-Net vs. FNO (Stärken/Schwächen):
%   \parencite{lu_comprehensive_2022}        -- Systematischer Vergleich neural operators
%   \parencite{qin_toward_2024}              -- Spektralbias FNO: warum Fehler nahe Kristallrand
%   \parencite{wen_u-fnoenhanced_2022}       -- U-FNO als mögliche Verbesserung (Ausblick)
%   \parencite{raonic_convolutional_2023}    -- Convolutional Neural Operators als Alternative
%
% Generalisierungsgrenzen:
%   \parencite{morimoto_generalization_2022} -- Faktoren, die Out-of-Distribution-Performance begrenzen
%   \parencite{rana_scalable_2024}           -- Domain-Decomposition als Skalierungsstrategie
%   \parencite{sanchez-gonzalez_learning_nodate} -- GNN-basierte Alternativen für viele Partikel
%
% Physikalische Einschränkungen / Inductive Bias:
%   \parencite{richter-powell_neural_nodate} -- Divergenzfreie Repräsentationen
%   \parencite{zhu_physics-constrained_2019} -- Physics-Constrained DL als Alternative
%   \parencite{kovachki_neural_nodate}       -- Neural Operator-Theorie (Approximierbarkeit)
%
% Geowissenschaftlicher Kontext:
%   \parencite{martin_crystal_1988}          -- Geophysikalische Relevanz der Ergebnisse
%   \parencite{verhoeven_numerical_2009}     -- Numerische Vergleichsbasis
%   \parencite{patocka_settling_2020}        -- Weitere Referenz für Sedimentation
%
% Zukünftige Arbeiten:
%   \parencite{hu_applying_2023}             -- PINN für Partikelströmung: physik-getriebene Erweiterung
%   \parencite{zhu_bayesian_2018}            -- Unsicherheitsquantifizierung als nächster Schritt
%   \parencite{koehler_apebench_nodate}      -- APEBench: Benchmark für autoregressive Erweiterungen
%   \parencite{li_fourier_nodate}            -- Geo-FNO: Erweiterung auf nicht-rechteckige Domänen
% ============================================================
%
\chapter{Discussion}
\label{ch:discussion}


Hier soll eine Diskussion der Ergebnisse meiner Experimente stattfinden. Ich werde die Leistung meines Modells anhand der Metriken bewerten, die ich in Kapitel 6 beschrieben habe, und die Vorhersagen meines Modells mit den Referenzlösungen vergleichen. Es wird auch eine Diskussion darüber geben, welche Aspekte meiner Modellierung gut funktionieren und wo es noch Herausforderungen gibt. Ich werde die Ergebnisse in Bezug auf die verschiedenen Lernziele und Eingabekodierungen analysieren und versuchen, Muster oder Trends zu identifizieren, die auf die Stärken und Schwächen meines Modells hinweisen. Schließlich werde ich die Ergebnisse in den Kontext der bestehenden Literatur einordnen und mögliche Implikationen für zukünftige Arbeiten diskutieren.
Wobei ich sicherstellen muss, dass ich hier nicht einfach die Ergebnisse aus Kapitel 8 wiederhole, sondern wirklich eine tiefere Analyse und Interpretation der Ergebnisse durchführe. Es soll nicht nur darum gehen, was die Ergebnisse sind, sondern auch warum sie so sind und was sie für die Praxis und die Forschung bedeuten. Es könnte auch eine Diskussion darüber geben, welche Einschränkungen oder Unsicherheiten es bei meinen Ergebnissen gibt und wie diese berücksichtigt werden sollten. Insgesamt soll dieses Kapitel dazu beitragen, die Bedeutung meiner Ergebnisse zu verstehen und ihre Auswirkungen auf die Praxis und die Forschung zu diskutieren.