% ----------------------------------------------------------------------------
% Chapter 8: Results
% ----------------------------------------------------------------------------
%
% ============================================================
% QUELLENVORSCHLÄGE FÜR DIESES KAPITEL
% ============================================================
% Einordnung der Ergebnisse (U-Net vs. FNO Vergleich):
%   \parencite{lu_comprehensive_2022}        -- Fairer Vergleich neural operators auf PDE-Benchmarks
%   \parencite{wen_u-fnoenhanced_2022}       -- U-FNO: Hybridarchitektur als Vergleichspunkt
%   \parencite{qin_toward_2024}              -- Spektralbias des FNO nahe scharfer Grenzflächen
%   \parencite{raonic_convolutional_2023}    -- Convolutional Neural Operators als Referenz
%
% Generalisierung über Kristallzahlen:
%   \parencite{morimoto_generalization_2022} -- Generalisierungstechniken für NN in Strömungen
%   \parencite{rana_scalable_2024}           -- Skalierbarkeit CNN-Surrogate auf mehr Hindernisse
%   \parencite{thuerey_deep_2020}            -- Generalisierung über Geometrien (U-Net)
%
% Fehleranalyse / Spatial Error Patterns:
%   \parencite{takamoto_pdebench_nodate}     -- Benchmark-Metriken zum Vergleich
%   \parencite{ribeiro_deepcfd_nodate}       -- Fehleranalyse nach geometrischen Regionen
%   \parencite{chen_u-net_nodate}            -- Räumliche Fehlerverteilung bei U-Net-Surrogaten
%
% Einbettung in Geowissenschaften:
%   \parencite{martin_crystal_1988}          -- Erwartete Fehler in Interaktionsregimen
%   \parencite{verhoeven_numerical_2009}     -- Numerische Referenzlösungen für Kristallsedimentation
%   \parencite{patocka_settling_2020}        -- Weitere numerische Referenz für Sedimentation
% ============================================================
%
\chapter{Results}
\label{ch:results}

Hier sollen die Ergebnisse meiner Experimente präsentiert und diskutiert werden. Ich werde die Leistung meines Modells anhand der Metriken bewerten, die ich in Kapitel 6 beschrieben habe, und die Vorhersagen meines Modells mit den Referenzlösungen vergleichen. Es wird auch eine Diskussion darüber geben, welche Aspekte meiner Modellierung gut funktionieren und wo es noch Herausforderungen gibt. Ich werde die Ergebnisse in Bezug auf die verschiedenen Lernziele und Eingabekodierungen analysieren und versuchen, Muster oder Trends zu identifizieren, die auf die Stärken und Schwächen meines Modells hinweisen. Schließlich werde ich die Ergebnisse in den Kontext der bestehenden Literatur einordnen und möglicheImplikationen für zukünftige Arbeiten diskutieren.

Was hier rein soll:
- Vergleich der Vorhersagen meines Modells mit den Referenzlösungen
- Diskussion der Leistung meines Modells anhand der Metriken
- Analyse der Ergebnisse in Bezug auf die verschiedenen Lernziele und Eingabekodierungen
- Einordnung der Ergebnisse in den Kontext der bestehenden Literatur
- Diskussion der Implikationen für zukünftige Arbeiten