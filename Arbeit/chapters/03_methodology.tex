% ============================================================================
% KAPITEL 3: METHODOLOGIE
% ============================================================================

\chapter{Methodologie und experimenteller Aufbau}
\label{ch:methodology}

\section{Experimentelles Design}

Das experimentelle Design folgt einem systematischen Ansatz zur Komplexitätssteigerung:

\begin{enumerate}
    \item \textbf{Baseline-Etablierung}: Ein-Kristall Systeme als Referenz
    \item \textbf{Komplexitätssteigerung}: Schrittweise Erhöhung auf 2-15 Kristalle
    \item \textbf{Generalisierungstests}: Cross-Validation über Kristallanzahlen
\end{enumerate}

\section{Physics-Informed Training}
\label{sec:physics_informed}

\subsection{Adaptive Loss-Gewichtung}

Die Integration physikalischer Constraints erfolgte durch eine adaptive Gewichtungsstrategie:

\begin{equation}
\mathcal{L}_{\text{total}} = \mathcal{L}_{\text{MSE}} + \lambda(t) \cdot \mathcal{L}_{\text{physics}}
\end{equation}

\begin{table}[htbp]
\centering
\caption{Adaptive Gewichtungsstrategie für Physics-Informed Loss}
\label{tab:lambda_schedule}
\begin{tabular}{@{}lcc@{}}
\toprule
\textbf{Trainingsphase} & \textbf{Epochen} & \textbf{$\lambda$-Wert} \\
\midrule
Warm-up & 1-5 & 0.01 \\
Transition & 6-15 & 0.05 \\
Full Physics & 16+ & 0.10 \\
\bottomrule
\end{tabular}
\end{table}

\subsection{Kontinuitäts-Constraint}

Der physikalische Loss-Term enforced die Inkompressibilität:

\begin{lstlisting}[language=Julia, caption={GPU-kompatible Divergenz-Berechnung}]
function compute_divergence(velocity_pred)
    vx = velocity_pred[:, :, 1, :]
    vz = velocity_pred[:, :, 2, :]
    
    # Zentrale Differenzen
    dvx_dx[2:end-1, :, :] = (vx[3:end, :, :] - vx[1:end-2, :, :]) / 2.0
    dvz_dz[:, 2:end-1, :] = (vz[:, 3:end, :] - vz[:, 1:end-2, :]) / 2.0
    
    return dvx_dx + dvz_dz
end
\end{lstlisting}

\section{Datenaugmentierung}

Zur Verbesserung der Generalisierung wurde systematische Datenaugmentierung implementiert:

\begin{itemize}
    \item \textbf{Horizontale Spiegelung}: Mit korrekter Vorzeichenumkehr für $v_x$
    \item \textbf{Zirkuläre Verschiebung}: Kleine Translationen zur Positionsvarianz
    \item \textbf{Effektive Vergrößerung}: 500 Samples → 1000+ augmentierte Samples
\end{itemize}