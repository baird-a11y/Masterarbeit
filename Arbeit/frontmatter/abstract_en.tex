\begin{abstract}
This master's thesis investigates the generalization capability of UNet architectures for predicting flow fields in multi-crystal sedimentation systems. While traditional computational fluid dynamics (CFD) methods are computationally intensive for complex multi-particle systems, machine learning approaches offer the potential for significant acceleration.

This exploratory study systematically evaluates whether UNet networks trained on a fixed number of crystals (e.g., 10 crystals) can successfully generalize to arbitrary other crystal configurations (1--15 crystals). Using LaMEM simulations as ground truth, controlled experiments are conducted where the model trained on $N$ crystals is evaluated across the entire spectrum from 1 to $N$ crystals. The results demonstrate [placeholder for actual findings], thereby characterizing both the possibilities and fundamental limitations of UNet generalization across different particle numbers.

This work contributes methodologically to physics-informed machine learning research and provides practical insights for geoscientific modeling applications.
\end{abstract}