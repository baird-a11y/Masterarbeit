\documentclass[aspectratio=169]{beamer}

% -------------------------------------------------
% Theme (calm & scientific)
% -------------------------------------------------
\usetheme{Madrid}
\usecolortheme{default}
\setbeamertemplate{navigation symbols}{}

% -------------------------------------------------
% Packages
% -------------------------------------------------
\usepackage{amsmath}
\usepackage{graphicx}
\usepackage{hyperref}
\usepackage{caption}
\captionsetup{font=scriptsize} % Macht alle Captions kleiner

% -------------------------------------------------
% Metadata
% -------------------------------------------------
\title{Machine Learning Surrogates for Viscous Flow}
\subtitle{Master’s Thesis Status Talk}
\author{Paul Baselt}
\institute{Computational Sciences – Geophysics}
\date{\today}

\begin{document}

% =================================================
\begin{frame}
  \titlepage
\end{frame}

% =================================================
\begin{frame}{Outline}
\begin{itemize}
  \item Motivation and challenge
  \item Machine learning approach
  \item Results and outlook
\end{itemize}
\end{frame}

% =================================================
\section{Motivation}

\begin{frame}{Motivation}
\begin{itemize}
    \item Crystal sedimentation is a key process in magmatic systems
    \item Crystals interact with viscous magma and with each other
    \item Even at very low Reynolds numbers, interactions are long-ranged
    \item Geometry strongly controls the resulting flow field
\end{itemize}

\vspace{0.2cm}
\begin{figure}
    \centering
    \includegraphics[width=0.6\textwidth]{Arbeit/Bilder/750px-How do different igneous rocks form from one original supply of magma-Fractional_crystallization.jpg}
    \caption{Schematic diagrams showing the principles behind fractional crystallisation in a magma. While cooling, the magma evolves in composition because different minerals crystallize from the melt. At the bottom of the magma reservoir, a cumulate rock forms.(https://www.geologyin.com/2014/11/magma-origination.html)}
    \end{figure}
\end{frame}

\begin{frame}{Why This Is a Hard Problem}
\begin{itemize}
    \item Stokes flow is linear but globally coupled
    \item Multiple crystals lead to non-trivial interaction patterns
    \item Small geometric changes can strongly affect the solution
\end{itemize}
\end{frame}

\begin{frame}{Computational Challenge}
\begin{itemize}
    \item Fully resolved Stokes simulations are expensive
    \item Cost increases rapidly with crystal number
    \item Large parameter studies become infeasible
\end{itemize}

\vspace{0.5cm}

\textbf{Goal:}\\
Investigate whether fast, physically informed surrogate models can generalize to increasing geometric complexity.
\end{frame}

% =================================================
\section{Research Question}

\begin{frame}{Research Question}
\begin{block}{Central Question}
How well can neural networks generalize flow-field predictions as geometric
complexity increases?
\end{block}

\vspace{0.3cm}

\begin{itemize}
    \item Not interpolation between similar cases
    \item Focus on unseen crystal configurations
\end{itemize}
\end{frame}


% =================================================
\section{Machine Learning Approach}

\begin{frame}{Learning Strategy}
\begin{itemize}
    \item Direct velocity prediction
    \item Stream function prediction
    \item Residual learning relative to analytical solution
\end{itemize}

\vspace{0.4cm}

\textbf{Current focus:}\\
Stream function prediction
\end{frame}

\begin{frame}{Data}
\begin{itemize}
    \item Generated using LaMEM
    \item 2D domain: $256 \times 256$ grid
\end{itemize}

\vspace{0.3cm}

\begin{itemize}
    \item Training data (current stage): 1--10 crystals
    \item Randomized positions (evaluation on unseen geometries)
    \item Fixed material parameters
\end{itemize}
\end{frame}

\begin{frame}{Input Representation}
\begin{itemize}
    \item Binary crystal mask
    \item Signed distance field in radius units: $SDF_R = \frac{d - R}{R}$
    \item Radius of the nearest crystal ($R$) as explicit channel
    \item Normalized $x$-coordinate
    \item Normalized $z$-coordinate
\end{itemize}

\vspace{0.3cm}
\end{frame}

\begin{frame}{Network Architecture}
\begin{itemize}
    \item U-Net architecture
    \item Encoder--decoder with skip connections
    \item Four encoder blocks
    \item Bottleneck
    \item Four decoder blocks
\end{itemize}

\vspace{0.3cm}

\begin{itemize}
    \item Strided convolutions instead of max pooling
\end{itemize}
\end{frame}

\begin{frame}{Training Setup}
\begin{itemize}
    \item 1000 - 2000 samples per configuration
    \item 100 - 300 training epochs
    \item Fixed learning rate and batch size
    \item Adam optimizer
\end{itemize}

\vspace{0.3cm}

\begin{itemize}
    \item Evaluation only on unseen geometries
\end{itemize}
\end{frame}

% =================================================
\section{Results}

\begin{frame}{Results Overview}
\begin{itemize}
  \item Single-crystal baseline
  \item Generalization to multiple crystals
  \item Error structure and limitations
\end{itemize}
\end{frame}

\begin{frame}{Results: Single Crystal}
\begin{itemize}
    \item Stream function predicted with high accuracy
    \item Relative $L^2$ error $\sim 10^{-2}$
\end{itemize}

\vspace{0.3cm}

\begin{itemize}
    \item Velocity errors larger due to differentiation (high-pass amplification)
\end{itemize}
\end{frame}

\begin{frame}{Visualization: One Crystal -- Stream Function}
\centering
\includegraphics[width=0.9\textwidth]{Arbeit/Bilder/sample_0002_psi.png}
\end{frame}

\begin{frame}{Visualization: One Crystal -- Velocity Magnitude}
\centering
\includegraphics[width=0.9\textwidth]{Arbeit/Bilder/sample_0002_speed_frompsi.png}
\end{frame}

\begin{frame}{Error Distance to Center}
\centering
\includegraphics[width=0.6\textwidth]{Arbeit/Bilder/error_vs_center_distance_by1.png}
\end{frame}

\begin{frame}{Error Distance to Upper Corner}
\centering
\begin{tabular}{cc}
    \includegraphics[width=0.45\textwidth]{Arbeit/Bilder/error_vs_corner3_distance_byN.png} &
    \includegraphics[width=0.45\textwidth]{Arbeit/Bilder/error_vs_corner4_distance_byN.png}
\end{tabular}
\end{frame}

\begin{frame}{Error Distance to Lower Corner}
\centering
\begin{tabular}{cc}
    \includegraphics[width=0.45\textwidth]{Arbeit/Bilder/error_vs_corner1_distance_byN.png} &
    \includegraphics[width=0.45\textwidth]{Arbeit/Bilder/error_vs_corner2_distance_byN.png}
\end{tabular}
\end{frame}

\begin{frame}{Generalization: Two Crystals -- $\psi$}
\centering
\includegraphics[width=0.9\textwidth]{Arbeit/Bilder/sample_0015_psi.png}
\end{frame}

\begin{frame}{Generalization: Two Crystals -- $|v|$}
\centering
\includegraphics[width=0.9\textwidth]{Arbeit/Bilder/sample_0015_speed_frompsi.png}
\end{frame}

\begin{frame}{First Generalization: Three Crystals -- $\psi$}
\centering
\includegraphics[width=0.9\textwidth]{Arbeit/Bilder/sample_0021_psi.png}
\end{frame}

\begin{frame}{First Generalization: Three Crystals -- $|v|$}
\centering
\includegraphics[width=0.9\textwidth]{Arbeit/Bilder/sample_0021_speed_frompsi.png}
\end{frame}

\begin{frame}{Results: Multiple Crystals}
\begin{itemize}
    \item Stream function error remains stable from 1 to 3 crystals
    \item Dominant interaction patterns are captured
    \item Some interaction regions remain challenging
\end{itemize}
\end{frame}

\begin{frame}{Visualization: One Crystal}
\centering
\includegraphics[width=0.9\textwidth]{Arbeit/Bilder/sample_0001_psi.png}
\end{frame}

\begin{frame}{Visualization: One Crystal}
\centering
\includegraphics[width=0.9\textwidth]{Arbeit/Bilder/sample_0001_speed_frompsi.png}
\end{frame}

\begin{frame}{First Generalization: Two Crystals}
\centering
\includegraphics[width=0.9\textwidth]{Arbeit/Bilder/sample_0012_psi.png}
\end{frame}

\begin{frame}{First Generalization: Two Crystals}
\centering
\includegraphics[width=0.9\textwidth]{Arbeit/Bilder/sample_0012_speed_frompsi.png}
\end{frame}

\begin{frame}{First Generalization: Three Crystals}
\centering
\includegraphics[width=0.9\textwidth]{Arbeit/Bilder/sample_0022_psi.png}
\end{frame}

\begin{frame}{First Generalization: Three Crystals}
\centering
\includegraphics[width=0.9\textwidth]{Arbeit/Bilder/sample_0022_speed_frompsi.png}
\end{frame}

\begin{frame}{First Generalization: Ten Crystals}
\centering
\includegraphics[width=0.9\textwidth]{Arbeit/Bilder/sample_0092_psi.png}
\end{frame}

\begin{frame}{First Generalization: Ten Crystals}
\centering
\includegraphics[width=0.9\textwidth]{Arbeit/Bilder/sample_0092_speed_frompsi.png}
\end{frame}

% (Optional duplicate plots kept as in your original deck)
\begin{frame}{Error Distance to Center}
\centering
\includegraphics[width=0.6\textwidth]{Arbeit/Bilder/error_vs_center_distance_byN_split.png}
\end{frame}

\begin{frame}{Error Distance to Upper Corner}
\centering
\begin{tabular}{cc}
    \includegraphics[width=0.45\textwidth]{Arbeit/Bilder/error_vs_corner3_distance_byN_split.png} &
    \includegraphics[width=0.45\textwidth]{Arbeit/Bilder/error_vs_corner4_distance_byN_split.png}
\end{tabular}
\end{frame}

\begin{frame}{Error Distance to Lower Corner}
\centering
\begin{tabular}{cc}
    \includegraphics[width=0.45\textwidth]{Arbeit/Bilder/error_vs_corner1_distance_byN_split.png} &
    \includegraphics[width=0.45\textwidth]{Arbeit/Bilder/error_vs_corner2_distance_byN_split.png}
\end{tabular}
\end{frame}

% =================================================
\section{Discussion}

\begin{frame}{Velocity Artefacts}
\begin{itemize}
    \item Velocity reconstructed from $\psi$ via derivatives
    \item Small grid-aligned errors are amplified by differentiation
\end{itemize}

\vspace{0.3cm}

\begin{itemize}
    \item Not a physical effect
    \item Numerical / architectural artefact
    %\item Gradient-based loss reduces these artefacts by enforcing derivative consistency
\end{itemize}
\end{frame}



% =================================================
\section{Outlook}

\begin{frame}{Outlook}
\begin{itemize}
    \item Residual learning relative to analytical solution
    \item Learn only interaction-driven corrections
    \item Reduced dynamic range
\end{itemize}
\end{frame}

% =================================================
\section{Backup}

\begin{frame}{Backup: Fourier Neural Operators}
\begin{itemize}
    \item Powerful for global operator learning
    \item Implicit translation invariance
\end{itemize}

\vspace{0.3cm}

\begin{itemize}
    \item Localized geometry and boundaries dominate here
    \item U-Net currently offers better control
\end{itemize}
\end{frame}

\end{document}
